\documentclass{article}
\usepackage{graphicx} % Required for inserting images
\usepackage{wrapfig}
\usepackage[table,xcdraw]{xcolor}
\usepackage{hyperref}
\usepackage[export]{adjustbox}

\usepackage{listings}
\usepackage{caption}
\usepackage{subfigure}
\usepackage{tikz}
\usepackage{enumitem}
\usepackage{multirow}
\usetikzlibrary{shapes, arrows.meta, positioning, trees}
\usepackage{forest}
\usepackage[bottom]{footmisc}
\usepackage{amsmath}
\usepackage{longtable}
\setcounter{secnumdepth}{4}
\setcounter{tocdepth}{4}

\begin{document}
    \tableofcontents
    \newpage
    
    \section{Che cos'è un'Organizzazione?}

    Le organizzazioni sono entità sociali orientate agli obiettivi, progettate come sistemi strutturati di attività coordinate, e collegate all'ambiente esterno. (Zara) Sono un mezzo per un fine, progettate per raggiungere obiettivi specifici attraverso il coordinamento delle persone e delle risorse.
    
    Un'organizzazione non è solo un edificio o un insieme di politiche; è costituita da persone e dalle loro relazioni. Esiste quando gli individui interagiscono per svolgere funzioni essenziali che aiutano a raggiungere obiettivi comuni.
    
    Nelle organizzazioni moderne, le interazioni esterne con clienti, fornitori, concorrenti e altri sono cruciali. Alcune organizzazioni collaborano anche con i concorrenti per condividere informazioni e tecnologie a beneficio reciproco.
    
    Le organizzazioni variano in dimensioni e struttura, con la distinzione principale tra imprese a scopo di lucro e organizzazioni non profit. Mentre le imprese mirano a generare profitto, le organizzazioni non profit si concentrano sulla creazione di un impatto sociale, spesso facendo affidamento su donazioni o fondi pubblici. I dirigenti delle imprese a scopo di lucro mirano ad aumentare i ricavi, mentre i dirigenti delle organizzazioni non profit lavorano per servire i clienti in modo efficiente, gestendo i costi per mantenere le operazioni e dimostrare l'efficienza delle risorse.
    
    \subsection{Dimensioni del Design Organizzativo}
    
    Le dimensioni strutturali forniscono etichette per le caratteristiche interne di un'organizzazione, consentendo il confronto e la misurazione.
    
    Esempio: Shizugawa → Alla scuola elementare Shizugawa, la creazione di regole e procedure dettagliate durante l'evacuazione per lo tsunami di Minamisanriku ha contribuito a mantenere l'ordine e a fornire comfort. Sei divisioni gestivano compiti quotidiani come cucina, pulizia e assistenza medica, garantendo operazioni fluide. Al contrario, la mancanza di autorità e struttura chiara durante il disastro di Deepwater Horizon ha esacerbato la crisi. La confusione su chi avesse l'autorità di agire ha portato al caos, con decisioni chiave, come l'emissione di un mayday o l'attivazione di uno spegnimento d'emergenza, ritardate a causa di responsabilità poco chiare e interruzioni della comunicazione.
    
    I fattori contingenti influenzano queste dimensioni e comprendono elementi come la dimensione, la tecnologia, l'ambiente, la cultura e gli obiettivi. Questi fattori possono essere complessi poiché influenzano sia l'organizzazione che il suo ambiente esterno.
    
    Le principali dimensioni strutturali includono:
    
    \begin{itemize}
        \item \textbf{Formalizzazione}: Si riferisce all'estensione della documentazione scritta all'interno di un'organizzazione, come procedure, descrizioni dei lavori, regolamenti e politiche. Misura la quantità di regole formali che definiscono i comportamenti e le attività.
        \item \textbf{Specializzazione}: Questa dimensione descrive come vengono suddivisi i compiti all'interno dell'organizzazione. Una alta specializzazione significa che i dipendenti svolgono compiti ristretti, mentre una bassa specializzazione consente un'ampia gamma di compiti all'interno dei ruoli.
        \item \textbf{Gerarchia dell'Autorità}: Definisce le relazioni di reporting e i span di controllo, determinando se l'organizzazione ha una gerarchia alta (molti livelli) o corta (pochi livelli).
        \item \textbf{Complessità}: La complessità si riferisce al numero di dipartimenti o attività distinti, misurati verticalmente (livelli di gerarchia), orizzontalmente (numero di dipartimenti), e spazialmente (distribuzione geografica).
        \item \textbf{Centralizzazione}: Questa dimensione indica dove risiede l'autorità decisionale. Le organizzazioni centralizzate hanno decisioni prese ai livelli più alti, mentre le organizzazioni decentralizzate spingono la decisione verso livelli inferiori.
    \end{itemize}
    
    I principali fattori contingenti delle organizzazioni includono dimensioni, tecnologia organizzativa, ambiente esterno, obiettivi e strategia, cultura organizzativa.
    
    \begin{itemize}
        \item \textbf{La dimensione} è tipicamente misurata dal numero di dipendenti, sebbene altri indicatori come le vendite totali o i beni posseduti possano anche riflettere la scala dell'organizzazione.
        \item \textbf{La Tecnologia Organizzativa} comprende gli strumenti, le tecniche e i processi utilizzati per convertire gli input in output, plasmando il modo in cui l'organizzazione produce beni e servizi.
        \item \textbf{L'Ambiente} include tutti gli elementi esterni che influenzano l'organizzazione, come industria, governo, clienti, fornitori e il settore finanziario, con altre organizzazioni che spesso sono quelle più influenti.
        \item \textbf{Gli Obiettivi e la Strategia} definiscono lo scopo dell'organizzazione, l'approccio competitivo e le azioni strategiche.
        \item \textbf{La Cultura} consiste in valori, credenze, comprensioni e norme condivise tra i dipendenti, che favoriscono la coesione e possono enfatizzare etica, impegno dei dipendenti ed efficienza.
    \end{itemize}
    
    Le cinque dimensioni strutturali e i cinque fattori contingenti sono interconnessi, con specifici fattori contingenti che influenzano i livelli ideali di specializzazione, formalizzazione e altri aspetti strutturali.
    
    \textbf{Esempio: Valve} → Valve Software opera con una struttura unica e piatta, dove non ci sono capi e i dipendenti hanno una notevole flessibilità e partecipazione nel processo decisionale. Sebbene questo favorisca la creatività e l'innovazione, non tutti si adattano al modello "senza struttura", portando alcuni a lasciare per ruoli più tradizionali. I team prendono le decisioni di assunzione e la leadership emerge naturalmente nei progetti, con riunioni informali del team e rare licenziamenti. Al contrario, la struttura di Walmart è altamente gerarchica, con controllo dall'alto verso il basso e procedure rigide, che portano a maggiore efficienza e prestazioni costanti, ma minore autonomia dei dipendenti nel processo decisionale.
    
    Ad esempio, una grande dimensione dell'organizzazione, una tecnologia routinaria e un ambiente stabile portano spesso a livelli più elevati di formalizzazione, specializzazione e centralizzazione.
    
    L'\textbf{efficienza} misura quanto bene un'organizzazione utilizza le risorse (ad esempio, materie prime, denaro e lavoro) per raggiungere i suoi obiettivi. È un riflesso della relazione tra input e output, mirando a ridurre al minimo gli sprechi di risorse mantenendo i risultati desiderati.
    
    L'\textbf{efficacia} va oltre l'efficienza e si concentra sull'obiettivo più ampio di raggiungere gli obiettivi dell'organizzazione. Un'organizzazione è considerata efficace quando soddisfa con successo i suoi obiettivi definiti, che possono riguardare il successo finanziario, la posizione di mercato o la soddisfazione del cliente.
    
    L'approccio degli \textbf{stakeholder} all'efficacia adotta una visione olistica, valutando quanto bene un'organizzazione soddisfa i bisogni e le aspettative di vari stakeholder, sia interni (ad esempio, dipendenti, manager) che esterni (ad esempio, clienti, investitori). Ad esempio:
    
    \begin{itemize}
        \item I clienti cercano prodotti e servizi di alta qualità a prezzi competitivi.
        \item I dipendenti desiderano una retribuzione equa, buone condizioni di lavoro e opportunità di crescita.
        \item Gli azionisti danno priorità a ritorni finanziari elevati sugli investimenti.
    \end{itemize}
    
    Considerando gli interessi diversificati di questi gruppi, l'approccio degli stakeholder fornisce una valutazione più completa dell'efficacia organizzativa, evidenziando l'importanza di bilanciare diverse priorità per ottenere il successo a lungo termine.
    
    \subsection{The Evolution of Organization Design}
Il design organizzativo fornisce un quadro per comprendere come le persone e le risorse sono organizzate per raggiungere obiettivi specifici, consentendo ai manager di analizzare i modelli nel comportamento e nel design organizzativo. Questa analisi più profonda aiuta a migliorare l'efficienza, l'efficacia e la qualità. La ricerca sul design organizzativo evidenzia anche come il design e le pratiche di gestione evolvano con i cambiamenti sociali.

La \textbf{gestione scientifica}, sviluppata da Frederick Taylor, si concentra sull'efficienza attraverso procedure standardizzate, ruoli di lavoro strutturati e pianificazione precisa da parte dei manager. Questo approccio ha influenzato significativamente le pratiche di gestione moderne, enfatizzando la produttività e la stabilità.

L'approccio dei \textbf{principi amministrativi} ha esteso la gestione classica concentrandosi sull'organizzazione complessiva, introducendo concetti come l'unità di comando. Ha promosso sistemi burocratici per migliorare la produttività, ma ha trascurato i fattori umani e sociali all'interno delle organizzazioni.

Un design organizzativo efficace deve tenere conto dei fattori contingenti, come la dimensione, la tecnologia, l'ambiente, gli obiettivi e la cultura, poiché non esiste una soluzione unica per tutte le situazioni. La \textbf{teoria delle contingenze} afferma che il miglior design dipende dalle circostanze uniche di un'organizzazione, come l'uso di una struttura burocratica in ambienti stabili o un design flessibile in aziende ad alta tecnologia e in ambienti incerti.

\subsection {The Contrast of Organic and Mechanistic Designs}
Le organizzazioni possono essere collocate su un continuum che va dal design meccanico al design organico.

Un \textbf{design meccanico} presenta regole rigide, procedure formali e decisioni centralizzate, focalizzandosi sull'efficienza. È adatto per ambienti stabili che danno priorità alla coerenza e al controllo.

Un \textbf{design organico} è più flessibile, decentralizzato e adattivo, con una struttura e regole meno formali. Enfatizza l'empowerment dei dipendenti e l'apprendimento, rendendolo ideale per ambienti dinamici che richiedono innovazione.

La scelta tra questi design dipende da fattori come la struttura, i compiti, la comunicazione e la gerarchia, che influenzano l'efficacia dell'organizzazione.

\textbf{Centralizzazione e Decentralizzazione}: In un design meccanico, il processo decisionale è centralizzato in cima, con il controllo detenuto dai dirigenti e i dipendenti che seguono le direttive. In un design organico, il processo decisionale è decentralizzato, con autorità e conoscenza distribuite a livelli inferiori, permettendo ai dipendenti di collaborare e prendere decisioni in modo indipendente.

\textbf{Compiti specializzati e ruoli empowerment}: I design meccanici assegnano compiti strettamente definiti agli individui, con attività specifiche per ogni dipendente. I design organici, invece, offrono ai dipendenti ruoli di empowerment, dando loro discrezione e responsabilità nell'utilizzare il loro giudizio per raggiungere gli obiettivi.

\textbf{Sistemi formali e informali}: I design meccanici si affidano a sistemi formali, regole e regolamenti per gestire la comunicazione e garantire l'adesione agli standard. I design organici presentano meno sistemi formali, consentendo una comunicazione e una condivisione delle informazioni più informali e flessibili.

\textbf{Comunicazione verticale e orizzontale}: Le organizzazioni meccaniche utilizzano principalmente la comunicazione verticale, dove le informazioni fluiscono su e giù nella gerarchia. Nelle organizzazioni organiche, la comunicazione è orizzontale, fluendo liberamente tra dipartimenti e livelli, consentendo decisioni più rapide.

\textbf{Gerarchia di autorità e lavoro collaborativo}: Le organizzazioni meccaniche hanno una gerarchia rigida con poca collaborazione tra i dipartimenti. Al contrario, le organizzazioni organiche enfatizzano il lavoro collaborativo, i flussi di lavoro orizzontali e la cooperazione interdipartimentale per risolvere i problemi, con team autonomi al centro.

\subsection{The Emerging Boss-less Design Trend}
L'ascesa del lavoro basato sulla conoscenza, dove le competenze e le idee generano valore, sta spingendo le organizzazioni verso la decentralizzazione. In tali organizzazioni, l'esperienza è distribuita e i dipendenti a tutti i livelli contribuiscono con idee. Risposte rapide ai cambiamenti ambientali e alle necessità dei clienti, insieme alla condivisione delle informazioni tramite la tecnologia, rendono ancora più necessaria la decentralizzazione. Di conseguenza, i livelli gerarchici possono rallentare il processo decisionale e aumentare i costi.

Alcune organizzazioni hanno adottato un design estremamente organico, \textbf{"senza capo"}, dove non esistono titoli di lavoro, anzianità o manager, e i dipendenti collaborano in modo paritario.

In un ambiente di lavoro senza capo (Morning Star), non vengono dati ordini e la responsabilità ricade sul cliente e sul team, non su un manager. Questa struttura può portare a una maggiore flessibilità e decisioni più rapide. Tuttavia, presenta anche sfide, come la necessità di formazione continua per i dipendenti e una cultura che supporti il sistema non gerarchico. Sebbene i costi generali possano essere più bassi, mantenere un ambiente senza capo richiede un significativo investimento nello sviluppo e nell'impegno dei dipendenti.

\subsection{Frameworks}

Le organizzazioni sono sistemi complessi composti da molteplici sottosistemi, analizzati a diversi livelli. A livello individuale, i dipendenti rappresentano gli elementi fondamentali, paragonabili alle cellule di un sistema biologico. Il livello successivo riguarda i gruppi o dipartimenti, nei quali gli individui collaborano per raggiungere obiettivi condivisi. A livello organizzativo, questi gruppi formano una struttura coesa. Oltre l'organizzazione stessa, esiste un insieme inter-organizzativo che comprende altre aziende con cui l'organizzazione interagisce.  

\textbf{Il comportamento organizzativo} adotta un approccio micro, concentrandosi sugli individui all'interno delle organizzazioni. Esamina fattori come motivazione, leadership, personalità e differenze cognitive ed emotive tra le persone, con l'obiettivo di comprendere come queste caratteristiche influenzino le dinamiche organizzative. L'analisi del comportamento organizzativo è cruciale per migliorare il coinvolgimento e la produttività dei dipendenti, contribuendo al successo complessivo dell'organizzazione.  

Al contrario, \textbf{la teoria e il design organizzativo} studiano l'organizzazione nel suo complesso, analizzandone la struttura, i dipartimenti e le interrelazioni, valutando il modo in cui questi elementi collaborano tra loro. Questo approccio macro è particolarmente rilevante per il top e il middle management, responsabili della definizione degli obiettivi, dello sviluppo delle strategie, dell'interpretazione dell'ambiente esterno e della determinazione della struttura organizzativa complessiva. Il middle management si occupa inoltre di garantire l'allineamento dei propri dipartimenti con l'organizzazione più ampia, gestendo sfide come i conflitti interdipartimentali e il flusso di informazioni. Il lower management si concentra maggiormente sulle operazioni quotidiane, sulla gestione dei dipendenti e sull'efficienza nell'esecuzione dei compiti.  


\section{Il Ruolo della Direzione Strategica nella Progettazione Organizzativa}

Gli obiettivi definiscono i risultati desiderati e guidano gli sforzi organizzativi, mentre i dirigenti di alto livello stabiliscono la direzione generale per raggiungere questi obiettivi. Il top management definisce gli obiettivi, la strategia e la struttura dell'organizzazione per rispondere a un ambiente dinamico, mentre i manager di livello intermedio adattano questi elementi ai propri dipartimenti.  

\paragraph{Esempio: Kroger}
Kroger sta rispondendo alla concorrenza e ai cambiamenti nelle esigenze dei consumatori ampliando la propria offerta di prodotti biologici, lanciando servizi di consegna a domicilio e investendo in tecnologia. L'azienda punta a fornire una vasta gamma di prodotti, inclusi marchi privati, e a ridurre gli sprechi alimentari con iniziative come \textit{Pickuliar Picks}. La strategia di Kroger comprende anche l'espansione della sua piattaforma di vendita diretta ai consumatori e la fusione con Home Chef per migliorare la comodità e la qualità dei prodotti freschi.  

\subsection{Il Processo di Definizione della Direzione Strategica}

Il processo di definizione della direzione strategica inizia con un'analisi delle opportunità e delle minacce esterne, concentrandosi sui cambiamenti, sull'incertezza e sulle risorse disponibili. I manager valutano anche i punti di forza e di debolezza interni per identificare competenze distintive rispetto ai concorrenti. Questa analisi SWOT (\textit{Strengths, Weaknesses, Opportunities, and Threats}) aiuta i leader a comprendere i fattori che influenzano le prestazioni dell'organizzazione.  

La \textbf{pianificazione degli scenari} è utilizzata per guidare la direzione strategica attraverso l'analisi delle tendenze e delle potenziali interruzioni. I manager immaginano diversi scenari futuri per prepararsi a incertezze come eventi climatici estremi o guasti ai sistemi. Per ogni fattore, vengono creati da due a cinque scenari, dai più ottimistici ai più pessimistici, per permettere all'organizzazione di rispondere in modo più efficace alle condizioni future.  

Una volta definita la direzione strategica, il passo successivo è stabilire l’\textbf{intento strategico}, che include la missione e gli obiettivi ufficiali dell'organizzazione. Questa fase assicura un allineamento tra le opportunità esterne e i punti di forza interni, ponendo le basi per lo sviluppo di obiettivi operativi e strategie per realizzare la missione.  

Segue la fase di \textbf{progettazione organizzativa}, in cui strategie e obiettivi vengono implementati attraverso decisioni riguardanti la struttura, i sistemi di controllo, la tecnologia, le risorse umane, la cultura e le connessioni con altre organizzazioni. I manager devono decidere se privilegiare l’innovazione e l’apprendimento (struttura organica) oppure l’efficienza e il controllo (struttura meccanicistica).  

La progettazione organizzativa è strettamente legata all’intento strategico. Le strutture esistenti influenzano la formulazione della strategia, ma nuovi obiettivi possono richiedere cambiamenti strutturali per adattarsi alle evoluzioni dell'ambiente.  

L'ultima fase consiste nella \textbf{valutazione dell'efficacia}, che analizza il raggiungimento degli obiettivi, l'efficienza nell'uso delle risorse e l'allineamento con gli obiettivi strategici. Le metriche di performance forniscono un feedback essenziale per aiutare i manager a perfezionare le strategie e a stabilire nuovi obiettivi, garantendo un miglioramento continuo e un'elevata capacità di adattamento.  

\subsection{Lo Scopo Organizzativo}

Le organizzazioni perseguono diversi tipi di obiettivi, ognuno con una funzione specifica. Per ottenere il successo a lungo termine, questi obiettivi devono essere allineati sotto un’intenzione strategica unitaria, che concentra risorse e sforzi verso un obiettivo chiaro. L'intento strategico fornisce direzione e garantisce che tutte le attività siano in linea con la missione centrale dell'organizzazione.  

Tre componenti essenziali dell’intento strategico sono:  

\begin{itemize}
    \item \textbf{La missione}: definisce lo scopo fondamentale dell’organizzazione, i suoi valori e i suoi obiettivi principali. Essa chiarisce la ragione dell’esistenza dell'organizzazione e i risultati che si propone di ottenere, spesso focalizzandosi su valori, mercati e bisogni dei clienti. La dichiarazione di missione è anche un importante strumento di comunicazione, utile per trasmettere gli obiettivi aziendali agli stakeholder.  
    \item \textbf{L'intento strategico}: mira a ottenere un vantaggio competitivo sostenibile, che differenzi l'organizzazione dai suoi concorrenti e le consenta di soddisfare le esigenze dei clienti in modo unico. I manager di successo monitorano sia le tendenze esterne che le capacità interne per identificare nuove opportunità e sviluppare strategie che rafforzino la posizione dell'organizzazione in un ambiente in continua evoluzione.  
    \item \textbf{La competenza distintiva}: rappresenta una forza unica che distingue un'organizzazione dalle altre. Può riguardare ambiti come un'elevata specializzazione tecnologica, un servizio clienti eccezionale o processi operativi altamente efficienti. Le competenze distintive rappresentano le capacità che l’organizzazione esegue in modo eccellente e costituiscono la base della sua strategia.  
\end{itemize}

\subsection{Obiettivi Operativi}
La missione e gli obiettivi complessivi dell’organizzazione costituiscono la base per lo sviluppo degli obiettivi operativi, che definiscono gli obiettivi specifici che un’organizzazione cerca di raggiungere attraverso le sue operazioni quotidiane. Questi obiettivi sono focalizzati su risultati misurabili a breve termine, guidando le attività quotidiane e le decisioni all’interno dei dipartimenti. Gli obiettivi operativi tipici includono obiettivi di prestazione, risorse, mercato, sviluppo dei dipendenti, produttività e innovazione.

\begin{itemize}
    \item \textbf{Prestazioni Generali}: Le organizzazioni for-profit misurano le prestazioni attraverso la redditività, spesso utilizzando metriche come il reddito netto, gli utili per azione o il ritorno sugli investimenti. Oltre alla redditività, altri obiettivi di prestazione includono la crescita, che si riferisce all’aumento delle vendite o dei profitti nel tempo, e il volume, che riguarda le vendite totali o il numero di prodotti o servizi consegnati. Questi indicatori aiutano a valutare il successo complessivo e la salute di un’organizzazione. Le organizzazioni non profit, che non danno priorità alla redditività, si concentrano sugli obiettivi di erogazione del servizio, assicurandosi di fornire valore all’interno dei livelli di spesa specificati. Crescita e volume sono comunque rilevanti per le organizzazioni non profit, in particolare per quanto riguarda l’espansione dei servizi, il raggiungimento di un numero maggiore di clienti o l’aumento dell’impatto nella comunità.
    
    \item \textbf{Risorse}: Gli obiettivi legati alle risorse riguardano l’acquisizione delle risorse materiali e finanziarie necessarie, come l’ottenimento di finanziamenti, l’approvvigionamento di materie prime più economiche o il reclutamento di talenti. Per le organizzazioni non profit, gli obiettivi delle risorse possono riguardare il reclutamento di volontari o l’espansione delle fonti di finanziamento.
    
    \item \textbf{Mercato}: Gli obiettivi di mercato si concentrano sulla quota di mercato o sulla posizione desiderata, solitamente gestiti dai team di marketing e vendite. Questi obiettivi mirano ad espandere la portata o l’influenza di un’organizzazione. Per le organizzazioni non profit, mirano ad ampliare il loro impatto o la loro erogazione di servizi in un settore specifico.
    
    \item \textbf{Sviluppo dei Dipendenti}: Lo sviluppo dei dipendenti include formazione, promozione, sicurezza e crescita dei dipendenti. Obiettivi forti in quest’area promuovono culture lavorative positive e migliorano le prestazioni del dipartimento, poiché i dipendenti acquisiscono le competenze necessarie. Le organizzazioni stanno dando sempre più importanza all’equilibrio tra lavoro e vita privata, oltre allo sviluppo delle competenze.
    
    \item \textbf{Produttività}: Gli obiettivi di produttività si concentrano sul massimizzare il rendimento dalle risorse disponibili, misurato tramite metriche come il costo per unità o le unità prodotte per dipendente. Aiutano le organizzazioni a valutare e migliorare l’efficienza operativa.
    
    \item \textbf{Innovazione e Cambiamento}: Gli obiettivi di innovazione si concentrano sull’adattamento ai cambiamenti sviluppando nuovi servizi, prodotti o processi. Questi obiettivi garantiscono che un’organizzazione possa rimanere competitiva e rispondere ai cambiamenti del mercato.
\end{itemize}


 
\subsection{Conflitto di Obiettivi}
Priorità diverse possono portare a disaccordi tra i manager, come bilanciare lo sviluppo dei dipendenti con gli obiettivi di produttività, o l'innovazione con la redditività. Questi conflitti nascono dalla tensione tra il perseguimento degli obiettivi finanziari e il mantenimento di valori come la privacy o la responsabilità sociale. Quando gli obiettivi e i valori si scontrano, i manager devono navigare attraverso queste tensioni mediante la negoziazione e il compromesso per determinare la migliore linea d'azione per l’organizzazione.

\subsection{L'Importanza degli Obiettivi}
Gli obiettivi ufficiali e gli obiettivi operativi servono a scopi distinti all’interno di un’organizzazione. Gli obiettivi ufficiali definiscono i valori e lo scopo complessivo dell’organizzazione. Gli obiettivi operativi sono più specifici, focalizzandosi sulle attività quotidiane e fornendo una direzione chiara per i dipendenti. Tuttavia, obiettivi mal gestiti possono portare a esiti negativi, come comportamenti non etici o pressioni indebite.

\subsubsection{Esempio: Wells Fargo}
Gli scandali di Wells Fargo sono stati alimentati da obiettivi di vendita aggressivi fissati dalla direzione, che hanno esercitato pressioni sui dipendenti per violare le regole, come aprire conti falsi e spingere prodotti non necessari. Nonostante gli avvertimenti etici, gli obiettivi elevati hanno portato a una condotta diffusa scorretta, con oltre 5.000 dipendenti coinvolti. La società inizialmente ha dato la colpa a azioni individuali, ma i dipendenti hanno sostenuto che gli obiettivi stessi fossero irrealistici senza comportamenti non etici.

Gli obiettivi fungono anche da linee guida per il comportamento e le decisioni dei dipendenti, assicurando che ci sia allineamento con i valori dell’organizzazione e le aspettative della società. Inoltre, gli obiettivi forniscono uno standard per valutare le prestazioni, offrendo parametri misurabili in aree come profitti, produttività, soddisfazione dei dipendenti, innovazione o reclami dei clienti.

\subsection{Le Strategie Competitive di Porter}
Porter ha identificato due strategie principali per migliorare la redditività di un’organizzazione e ridurre la vulnerabilità alla concorrenza: differenziazione e leadership a basso costo. Queste strategie aiutano le organizzazioni a decidere come posizionarsi nel settore per raggiungere la redditività e la sostenibilità.

\subsubsection{Esempio: Allegiant Air}
Allegiant Air si concentra su una strategia di \textbf{leadership a basso costo}, servendo piccole città non ben servite con voli per destinazioni turistiche. La compagnia aerea punta su opportunità di nicchia, come colmare le lacune lasciate da compagnie aeree più grandi in mercati in contrazione. Con un approccio senza fronzoli, tariffe basse di base e addebiti per servizi aggiuntivi, Allegiant mantiene la redditività mantenendo bassi i costi. L’azienda si è espansa anche sui mercati internazionali come il Messico, continuando a concentrarsi sull’efficienza dei costi e sui mercati regionali.

Una strategia di \textbf{differenziazione} comporta l’offerta di prodotti o servizi unici attraverso caratteristiche eccezionali, un servizio superiore o tecnologie innovative. Questo approccio mira a clienti che apprezzano la qualità più del prezzo, con il potenziale di alta redditività. Una strategia di \textbf{differenziazione} di successo riduce la concorrenza e la minaccia dei sostituti favorendo la fedeltà dei clienti. Tuttavia, richiede un investimento significativo in sviluppo del prodotto, pubblicità e talento.

La strategia di \textbf{leadership a basso costo} mira ad aumentare la quota di mercato minimizzando i costi rispetto ai concorrenti. Questa strategia si concentra sull'efficienza operativa, sulla riduzione dei costi e su un controllo rigoroso per produrre beni o servizi in modo più efficiente. Sebbene i leader a basso costo possano offrire prezzi più bassi, l'enfasi è sull'efficienza operativa piuttosto che sull'innovazione o sul prendere rischi. Mantenendo costi inferiori, le aziende possono offrire prezzi competitivi pur rimanendo redditizie. Questa strategia aiuta a proteggere la quota di mercato dai concorrenti e dai sostituti.


\section{Tipologia di Strategie di Miles e Snow}
L’obiettivo è allineare le caratteristiche interne, la strategia e il contesto esterno. Secondo questa tipologia, esistono quattro tipi di strategia: 
\begin{description}
    \item[\textbf{Prospector}] La strategia del \textbf{prospector} enfatizza l'innovazione, la presa di rischi e la ricerca di nuove opportunità per guidare la crescita. È adatta a ambienti dinamici in cui la creatività e la differenziazione sono prioritarie rispetto all’efficienza. Le organizzazioni che adottano questa strategia si concentrano sull’esplorazione continua e sull’espansione.
    
    \item[\textbf{Defender}] La strategia del \textbf{defender} si concentra sulla stabilità e sul controllo. Mira a mantenere la base di clienti esistente e l’efficienza interna, piuttosto che cercare innovazione o crescita. Questa strategia è efficace in settori stabili o in declino, dove l’obiettivo è produrre prodotti affidabili e di alta qualità evitando rischi.
    
    \item[\textbf{Analyzer}] La strategia dell’\textbf{analyzer} bilancia stabilità e innovazione. Le organizzazioni che adottano questa strategia mantengono operazioni efficienti in ambienti stabili, esplorando nel contempo nuove opportunità in mercati dinamici. Conservano le linee di prodotto esistenti mentre sviluppano nuove per la crescita.
    
    \item[\textbf{Reactor}] La strategia del \textbf{reactor} manca di un piano proattivo, con le organizzazioni che rispondono a minacce ed opportunità esterne in modo ad-hoc. Questo approccio reattivo può talvolta avere successo, ma spesso porta al fallimento quando non si allinea con le tendenze del mercato o le necessità dei consumatori.
\end{description}

La tipologia di Miles e Snow è stata ampiamente testata in settori come ospedali, università, banche e compagnie assicurative, mostrando un forte supporto per la sua efficacia pratica. La capacità dei manager di sviluppare e mantenere una strategia competitiva chiara è la chiave del successo di un’organizzazione, anche se molti manager affrontano difficoltà in quest’area.

\section{Come le Strategie Influenzano il Design dell'Organizzazione}
La scelta della strategia influisce direttamente sul design e sulla struttura di un’organizzazione. Un’azienda focalizzata sulla \textbf{leadership a basso costo} tende ad adottare un approccio meccanistico, enfatizzando l’efficienza, l’autorità centralizzata, il controllo stretto e le procedure standardizzate. I dipendenti svolgono compiti di routine con supervisione ravvicinata e autonomia decisionale limitata.

Al contrario, una strategia di \textbf{differenziazione} richiede una struttura più organica e flessibile che favorisca l’apprendimento e la sperimentazione. Essa implica un processo decisionale decentralizzato, una forte coordinazione orizzontale e un design organizzativo fluido e adattabile per supportare l’innovazione continua.

Nelle organizzazioni che seguono una strategia di \textbf{differenziazione} o del \textbf{prospector}, i dipendenti sono incoraggiati a lavorare direttamente con i clienti e a prendere rischi, essere creativi e innovare. Queste organizzazioni danno priorità alla ricerca, alla creatività e all’innovazione rispetto all’efficienza e alla standardizzazione.

La strategia del \textbf{defender}, simile all’approccio della \textbf{leadership a basso costo}, si concentra sull’efficienza, enfatizzando la stabilità e il controllo. La strategia dell’\textbf{analyzer} mescola caratteristiche di entrambi gli approcci, bilanciando l’efficienza nelle linee di prodotto stabili con flessibilità e apprendimento per nuovi prodotti.

Una strategia del \textbf{reactor}, invece, manca di una direzione chiara e di un approccio al design dell’organizzazione, lasciando l’azienda senza una strategia o struttura definita.


\subsection{Un Modello Integrato di Efficacia}
Il modello affronta queste complessità integrando molteplici indicatori di performance. È stato sviluppato per riconciliare diverse visioni sull’efficacia, combinando i criteri valutati da manager e ricercatori. Attraverso un’analisi esperta, identifica le dimensioni sottostanti dell’efficacia che riflettono i valori in competizione nella gestione, fornendo un quadro per bilanciare queste priorità.

\subsection{Indicatori}
Il modello incorpora due dimensioni principali: focus organizzativo e struttura. 
\begin{itemize}
    \item \textbf{Focus} si riferisce a se le priorità dell’organizzazione sono interne (focalizzate sul benessere dei dipendenti e sull’efficienza) o esterne (focalizzate sul successo dell’organizzazione nel suo ambiente).
    \item \textbf{Struttura} riguarda se l’organizzazione valorizza la stabilità (efficienza e controllo) o la flessibilità (apprendimento e adattabilità).
\end{itemize}
Combinando queste dimensioni si ottengono quattro approcci all’efficacia, ognuno con un distinto focus manageriale. Ad esempio, un focus esterno combinato con flessibilità crea un approccio di \textbf{open systems}, che dà priorità alla crescita e all’acquisizione di risorse attraverso l’adattabilità e le relazioni esterne positive.

Il modello dei valori competitivi esamina l’efficacia organizzativa utilizzando due dimensioni primarie: focus e struttura. 
\begin{itemize}
    \item \textbf{Focus} determina se i valori principali dell’organizzazione sono interni (centrati sul benessere dei dipendenti e sull’efficienza operativa) o esterni (focalizzati sul successo dell’organizzazione nel suo ambiente).
    \item \textbf{Struttura} riguarda se l’organizzazione enfatizza la stabilità (favorendo efficienza e controllo) o la flessibilità (dando priorità all’adattabilità e all’apprendimento).
\end{itemize}

Combinando queste dimensioni si ottengono approcci differenti all’efficacia. 

Quando un’organizzazione ha un \textbf{focus esterno} e una \textbf{struttura flessibile}, adotta un approccio di \textbf{sistemi aperti}. Qui, gli obiettivi principali sono la \textbf{crescita} e l'\textbf{acquisizione di risorse}, raggiunti attraverso l’adattabilità e le \textbf{relazioni esterne positive}, permettendo all’organizzazione di prosperare nel suo ambiente. Questo approccio è allineato con l’idea di sfruttare le \textbf{risorse} in modo efficace per una \textbf{crescita a lungo termine}.

L’enfasi sugli \textbf{obiettivi razionali} combina un \textbf{focus esterno} con il \textbf{controllo strutturale}, dando priorità alla \textbf{produttività}, all’\textbf{efficienza} e al \textbf{profitto}. Il suo obiettivo principale è raggiungere gli obiettivi di output attraverso una \textbf{pianificazione strutturata} e la \textbf{definizione degli obiettivi}, riflettendo un approccio \textbf{razionale} e \textbf{controllato} alla gestione.

L’enfasi sui \textbf{processi interni} si allinea con un \textbf{focus interno} e un \textbf{controllo strutturale}, mirando a creare un \textbf{ambiente organizzativo stabile} e ordinato. Essa enfatizza la \textbf{comunicazione efficiente}, la \textbf{gestione delle informazioni} e i \textbf{processi decisionali} per mantenere la posizione attuale dell’organizzazione.

L’enfasi sulle \textbf{relazioni umane} sottolinea un \textbf{focus interno} abbinato a una \textbf{struttura flessibile}, con un obiettivo principale di \textbf{sviluppo delle risorse umane}. Questo approccio supporta l’\textbf{autonomia} dei dipendenti, la \textbf{coesione}, il \textbf{morale} e la \textbf{formazione}, dando priorità alle \textbf{relazioni interne} e allo \textbf{sviluppo} rispetto alle preoccupazioni esterne.


\subsection{Fundamentals of Organization Structure}

\subsubsection{Organization Structure}

La struttura organizzativa è definita da tre componenti principali:
\begin{enumerate}
    \item Essa designa le posizioni formali, le relazioni di reporting e i livelli nella gerarchia, inclusa l'ampiezza di controllo dei manager.
    \item Essa implica il raggruppamento degli individui in dipartimenti e dei dipartimenti nell’organizzazione.
    \item Essa include sistemi per una comunicazione, coordinazione e integrazione efficaci tra i dipartimenti.
\end{enumerate}

Questi componenti affrontano sia gli aspetti verticali che orizzontali dell’organizzazione, con i primi due focalizzati sulla gerarchia strutturale e il terzo sulla promozione dell’interazione e della coordinazione tra i dipendenti.

La struttura organizzativa è rappresentata dal diagramma dell’organizzazione, che mostra visivamente le attività e i processi all’interno di un’organizzazione. Esso mostra le diverse parti dell’organizzazione, le loro interrelazioni e come le posizioni e i dipartimenti si inseriscono nella struttura complessiva.

Il concetto di diagramma organizzativo esiste da secoli, ma il suo uso in ambito aziendale è diventato più significativo durante la Rivoluzione Industriale, quando la complessità del lavoro e la necessità di una gestione migliore sono diventate più evidenti. Il diagramma aiuta a definire le responsabilità e le autorità all’interno dell’organizzazione.

\subsubsection{Information-Sharing Perspective on Structure}

L’organizzazione dovrebbe facilitare sia il flusso di informazioni verticale che orizzontale per raggiungere i suoi obiettivi. Un disallineamento tra la struttura e le necessità informative può portare a informazioni insufficienti o a tempo sprecato su dati irrilevanti, riducendo l’efficacia. Tuttavia, esiste un conflitto tra i meccanismi verticali e orizzontali: i legami verticali si concentrano sul controllo, mentre i legami orizzontali enfatizzano la coordinazione e la collaborazione, che tipicamente riducono il controllo.

\subsubsection{Centralized vs Decentralized}

Il livello al quale vengono prese le decisioni in un’organizzazione influenza il flusso delle informazioni. La centralizzazione colloca l’autorità decisionale in alto, mentre la decentralizzazione la spinge a livelli più bassi. Le organizzazioni possono scegliere tra un design meccanico, che enfatizza l’efficienza, la comunicazione verticale, il controllo e la decisione centralizzata, o un design organico, che privilegia la flessibilità, la comunicazione orizzontale, la coordinazione e la decisione decentralizzata. Il modello meccanico supporta compiti specializzati e una gerarchia rigida, mentre il modello organico incoraggia compiti condivisi, meno regole, comunicazione faccia a faccia e più team.

\textbf{Esempio: Toyota} → Toyota si è orientata verso la decentralizzazione dopo aver ricevuto critiche per il suo controllo centralizzato, specialmente durante i problemi di sicurezza e i richiami. L’azienda ha iniziato a delegare maggiore autorità decisionale ai manager regionali, in particolare per le problematiche di sicurezza nelle varie regioni. Sebbene alcune decisioni siano rimaste presso la sede centrale, questo cambiamento mirava a migliorare la reattività e il controllo qualità, consentendo di prendere decisioni più vicine all’azione.

Le organizzazioni spesso sperimentano per determinare il miglior equilibrio tra centralizzazione e decentralizzazione. Spostarsi verso la decentralizzazione può migliorare le prestazioni promuovendo l’autonomia e la responsabilità. Alcune aziende stanno decentralizzando per responsabilizzare i dipendenti e favorire un senso di appartenenza.

Tuttavia, la decentralizzazione ha dei rischi, poiché troppa decentralizzazione può portare a problemi, inclusi quelli etici e legali. Esiste spesso una tensione tra centralizzazione e decentralizzazione all’interno delle organizzazioni. I dirigenti di alto livello possono preferire la centralizzazione per il controllo e l’efficienza, mentre i manager aziendali preferiscono la decisione decentralizzata. I manager lavorano per trovare l’equilibrio ottimale tra il controllo verticale, la collaborazione orizzontale, la centralizzazione e la decentralizzazione in base alle loro necessità specifiche.

\subsubsection{Vertical Information Sharing}

Il design dell’organizzazione deve abilitare la comunicazione e la coordinazione per raggiungere gli obiettivi. I manager creano legami informativi per facilitare questo processo. I legami verticali, progettati per il controllo, coordinano le attività tra la parte superiore e quella inferiore dell’organizzazione. I dipendenti a livelli più bassi devono allinearsi agli obiettivi del livello superiore, mentre i dirigenti devono rimanere informati sulle attività a livelli inferiori. Dispositivi strutturali, come il rinvio gerarchico, le regole e i piani, e i sistemi informativi formali, stabiliscono legami verticali.

\begin{description}
    \item[Hierarchical referral] Fa salire i problemi non risolti ai livelli superiori della gerarchia, con le soluzioni comunicate nuovamente attraverso le linee organizzative.
    \item[Rules and plans] Forniscono coordinamento stabilendo risposte predefinite per problemi ripetitivi, riducendo la necessità di comunicazione diretta con i manager. Piani come i budget aiutano i dipendenti a operare all'interno di parametri stabiliti.
    \item[Vertical information systems] Migliorano la comunicazione con report, informazioni scritte e comunicazioni digitali, aumentando l'efficienza nel flusso delle informazioni. In risposta agli scandali aziendali, i manager stanno rafforzando i legami verticali per un migliore controllo, mentre i legami orizzontali sono altrettanto importanti per la collaborazione.
\end{description}


\subsubsection{Condivisione Orizzontale delle Informazioni e Collaborazione}

La comunicazione orizzontale supera le barriere tra i dipartimenti, favorendo la collaborazione per raggiungere gli obiettivi organizzativi. È fondamentale per compiti complessi che richiedono sforzi congiunti, come nelle collaborazioni militari e di intelligence, che si sono rivelate efficaci in operazioni ad alto rischio come l'operazione per catturare Bin Laden. Storicamente, i settori militare e di intelligence avevano interazioni limitate, ma la necessità di azioni coordinate ha dimostrato la forza della collaborazione interfunzionale.

\textbf{Esempio: AT\&T} → Dopo che AT\&T ha acquisito Time Warner, sono stati formati gruppi di lavoro per promuovere la collaborazione tra le divisioni. Questi gruppi hanno lavorato su progetti come lo sviluppo di un nuovo servizio video in abbonamento per competere con Netflix. L'iniziativa mirava a ridurre i conflitti interni e promuovere la cooperazione, spostandosi verso un approccio organizzativo più unificato.



Il \textit{linkaggio orizzontale} si riferisce alla comunicazione tra i dipartimenti. Nelle grandi organizzazioni, i meccanismi per la condivisione delle informazioni sono cruciali per una collaborazione e un processo decisionale efficaci. Una cattiva coordinazione può ritardare le risposte, quindi sono necessari dispositivi strutturali per migliorare la comunicazione orizzontale.

I sistemi informativi abilitano il linkaggio orizzontale consentendo a manager e dipendenti di condividere aggiornamenti e risolvere problemi. Questi sistemi facilitano la costruzione di relazioni e la coordinazione, rafforzando le connessioni e le performance organizzative.


I \textit{ruoli di collegamento} sono posizioni all'interno di un dipartimento che facilitano la comunicazione e la coordinazione tra i dipartimenti. Sono comuni tra dipartimenti come ingegneria e produzione, per garantire che lo sviluppo del prodotto sia allineato con le capacità di produzione, o tra ricerca e vendite per allineare i prodotti con le richieste dei clienti.


I \textit{comitati di lavoro} sono comitati temporanei formati per affrontare questioni specifiche che coinvolgono più dipartimenti. Ogni membro rappresenta il proprio dipartimento, assicurando una collaborazione diretta orizzontale. I comitati vengono sciolti una volta raggiunti gli obiettivi, contribuendo a ridurre il carico informativo sulla gerarchia verticale.


Gli \textit{integratori a tempo pieno} sono ruoli dedicati, come i responsabili di prodotto, progetto o brand, che sono incaricati di coordinare tra i dipartimenti per compiti complessi. A differenza dei collegamenti, gli integratori sono indipendenti dai dipartimenti che coordinano, gestendo compiti come vendite e pubblicità. Non hanno autorità formale, ma si affidano a competenze ed abilità interpersonali per promuovere la cooperazione e risolvere i conflitti.


I \textit{team interfunzionali} sono gruppi permanenti con membri provenienti da varie aree funzionali che lavorano insieme su progetti a lungo termine. Questi team sono cruciali per iniziative che richiedono una collaborazione continua, unendo competenze da ricerca, marketing e finanza. I team interfunzionali virtuali abilitano la collaborazione globale attraverso strumenti di comunicazione digitale, consentendo ai team di lavorare attraverso fusi orari e culture in uno spazio virtuale attivo 24 ore su 24.

\subsubsection{Coordinamento Relazionale}
Il \textit{coordinamento relazionale} è il più alto livello di coordinamento orizzontale, caratterizzato da comunicazione frequente, tempestiva e orientata alla risoluzione dei problemi, basata su obiettivi condivisi, conoscenza e rispetto reciproco. A differenza dei meccanismi formali, il coordinamento relazionale è radicato nella cultura di un'organizzazione e promuove una collaborazione senza soluzione di continuità tra i dipartimenti.

\textbf{Esempio: Southwest Airlines} → Southwest Airlines utilizza il coordinamento relazionale per migliorare le sue performance in termini di puntualità e soddisfazione del cliente. Piuttosto che assegnare colpe per i ritardi, l'azienda promuove il lavoro di squadra concentrandosi su obiettivi condivisi come la partenza puntuale e la sicurezza. I manager enfatizzano la collaborazione e forniscono supporto, con i supervisori che allenano i dipendenti, garantendo una stretta coordinazione tra i vari dipartimenti. Questo approccio aiuta la compagnia aerea a mantenere il tempo di rotazione più breve del settore.

Le caratteristiche chiave includono:
\begin{itemize}
    \item Condivisione libera delle informazioni e interazione continua tra i confini dei dipartimenti.
    \item Coordinamento guidato da relazioni positive piuttosto che da ruoli o regole formali.
\end{itemize}

I manager giocano un ruolo cruciale in:
\begin{itemize}
    \item Addestrare i dipendenti nella comunicazione e nella risoluzione dei conflitti.
    \item Costruire fiducia attraverso cura e credibilità.
    \item Promuovere obiettivi condivisi sopra i silo dipartimentali.
    \item Consentire flessibilità nelle regole di lavoro e premiare i successi del team.
    \item Creare ruoli interfunzionali per supportare la collaborazione.
    \item Mantenere spazi di controllo piccoli per consentire una supervisione più stretta e un mentoring.
\end{itemize}

Un alto coordinamento relazionale consente ai team di condividere informazioni e risolvere problemi senza fare affidamento su strutture formali o direttive.

\subsubsection{Attività di Lavoro Richieste}

I dipartimenti sono creati per svolgere compiti che sono strategicamente significativi per un'organizzazione. Man mano che le organizzazioni crescono e le loro operazioni diventano più complesse, i manager creano nuovi ruoli, dipartimenti o divisioni per affrontare le esigenze emergenti e realizzare ulteriori compiti di valore. Questa espansione consente all'organizzazione di adattarsi a nuove sfide e opportunità, garantendo il suo successo continuo e allineamento con gli obiettivi strategici.

\subsubsection{Relazioni di Reporting}

Una volta che le attività di lavoro e i dipartimenti sono definiti, il passo successivo è determinare come si inseriscono all'interno della gerarchia organizzativa. Le relazioni di reporting, o la catena di comando, sono rappresentate come linee verticali in un organigramma. Questa catena di comando stabilisce una linea continua di autorità, collegando tutte le persone e chiarendo chi riporta a chi. Definire i dipartimenti e creare le relazioni di reporting delinea come i dipendenti sono raggruppati all'interno dell'organizzazione.

\subsubsection{Opzioni per il Raggruppamento dei Dipartimenti}

Le opzioni per il raggruppamento dei dipartimenti includono il raggruppamento funzionale, divisionale, multifocale, a rete virtuale e a team di holacracy. Il raggruppamento dei dipartimenti influisce sui dipendenti allineandoli sotto un supervisore comune, condividendo risorse, favorendo la collaborazione e garantendo la responsabilità condivisa per le performance.

\begin{itemize}
    \item \textbf{Raggruppamento funzionale:} organizza i dipendenti in base a funzioni, processi o competenze simili.
    \item \textbf{Raggruppamento divisionale:} organizza i dipendenti in base ai prodotti o ai servizi che l'organizzazione produce.
    \item \textbf{Raggruppamento matriciale:} combina due o più metodi di raggruppamento strutturale, come la funzione e le divisioni di prodotto, per raggiungere più obiettivi contemporaneamente.
    \item \textbf{Raggruppamento a rete virtuale:} collega componenti o dipartimenti separati elettronicamente, permettendo la collaborazione e l'esecuzione dei compiti in diverse località.
    \item \textbf{Raggruppamento a team di holacracy:} utilizza team autogestiti per realizzare compiti o attività specifiche.
\end{itemize}

\subsection{Progettazioni Funzionali, Divisionali e Geografiche}

\subsubsection{Struttura Funzionale}

Una \textit{struttura funzionale}, o \textit{U-form}, raggruppa le attività in base a funzioni comuni, concentrando competenze e abilità all'interno di dipartimenti distinti. Funziona bene quando le priorità includono conoscenze specializzate, controllo verticale e efficienza, con poca necessità di coordinazione orizzontale.

I suoi vantaggi includono economie di scala, minimizzazione delle duplicazioni e sviluppo delle competenze all'interno dei dipartimenti. Tuttavia, gli svantaggi comprendono una lenta adattabilità ai cambiamenti, un sovraccarico nel processo decisionale ai livelli più alti, limitata innovazione a causa di una scarsa coordinazione interdipartimentale e un focus ristretto sugli obiettivi organizzativi da parte dei dipendenti.

\subsubsection{Struttura Funzionale con Linkaggi Orizzontali}

Sebbene le strutture funzionali siano efficaci e rimangano comuni, spesso faticano ad adattarsi all'ambiente frenetico di oggi. Le piccole organizzazioni possono fare affidamento sulla coordinazione informale, ma le organizzazioni più grandi generalmente richiedono meccanismi più forti per il linkaggio orizzontale per garantire la collaborazione e il coordinamento.

I manager possono migliorare la coordinazione orizzontale attraverso metodi come sistemi informativi, ruoli di collegamento, integratori a tempo pieno o project manager, comitati di lavoro, team e promuovendo il coordinamento relazionale. L'implementazione dei linkaggi orizzontali aiuta a superare le limitazioni delle strutture funzionali, come la lenta adattabilità e la scarsa collaborazione interdipartimentale.

\subsubsection{Struttura Divisionale}

Una \textit{struttura divisionale}, o \textit{M-form}, organizza divisioni separate responsabili per prodotti, servizi o centri di profitto specifici. Ogni divisione opera come un'unità autonoma con i propri dipartimenti funzionali, come R\&D, marketing e contabilità, adattati ai suoi output. Questa struttura è comunemente adottata da organizzazioni in crescita per gestire la crescente complessità.

\textbf{Esempio: Google e Alphabet} → Il CEO ha deciso di ristrutturare le varie attività di Google sotto una nuova società madre, chiamata Alphabet, per dare ai manager nelle aziende maggiore autonomia e mantenere le aziende innovative e adattabili. La divisione più grande rimane comunque Google.

Ogni divisione ha i suoi obiettivi e budget, contiene tutte le funzioni necessarie per eseguire i propri compiti e ha il proprio CEO e team di gestione. La struttura divisionale offre diversi punti di forza, tra cui maggiore flessibilità, adattabilità e coordinamento all'interno delle divisioni, rendendola ideale per ambienti dinamici. Migliora anche la visibilità del prodotto, consentendo alle divisioni di rispondere rapidamente ai cambiamenti del mercato e alle esigenze dei clienti. La decisione decentralizzata favorisce l'agilità e l'orientamento al cliente, in particolare nelle organizzazioni con più offerte.

Tuttavia, ci sono degli svantaggi. La struttura sacrifica le economie di scala a causa della duplicazione delle risorse, il che può aumentare i costi e ostacolare la ricerca approfondita. Le divisioni possono diventare isolate, complicando la coordinazione senza meccanismi orizzontali come i comitati di lavoro, portando a inefficienze e insoddisfazione del cliente. Il focus sulla ricerca applicata può anche ridurre l'attenzione sugli obiettivi più ampi dell'organizzazione, indebolendo la specializzazione tecnica.

Questi problemi sono amplificati negli ambienti internazionali a causa delle differenze geografiche, culturali e linguistiche. Per mitigare questi problemi, i manager devono implementare sistemi che promuovano la collaborazione orizzontale e la comunicazione efficace tra le divisioni.

\subsection{Relazioni Interorganizzative}

\subsubsection{Ecosistemi Organizzativi}

Le relazioni interorganizzative si riferiscono a transazioni di risorse, flussi e collegamenti duraturi tra due o più organizzazioni. Tradizionalmente considerate necessarie ma indesiderabili, queste relazioni venivano spesso viste come un compromesso per soddisfare le necessità organizzative in ambienti complessi e instabili.

Tuttavia, una prospettiva moderna, proposta da James Moore, vede le organizzazioni come parte di ecosistemi aziendali più ampi. Questi ecosistemi sono costituiti da organizzazioni interconnesse e dai loro ambienti, che attraversano i confini tradizionali dell'industria. Un approccio simile è quello della \textit{megacommunity}, dove aziende, governi e organizzazioni non profit collaborano tra settori per affrontare grandi sfide condivise come lo sviluppo energetico, la fame o il crimine informatico.

\subsubsection{La Competizione è Morta?}

Nessuna organizzazione può prosperare in modo indipendente nell'ambito di una competizione globale, dei progressi tecnologici e dei cambiamenti normativi. Le aziende ora operano in reti complesse, spesso collaborando in alcuni settori mentre competono in altri. La competizione tradizionale, dove le aziende agiscono come rivali indipendenti, si è evoluta verso l'interdipendenza, richiedendo supporto reciproco per il successo e la sopravvivenza.

\textbf{Esempio: Apple e Samsung} → rivali nelle guerre degli smartphone. Samsung guadagna dalle vendite degli iPhone, collaborando con Apple. Hanno lavorato insieme per un decennio per costruire chip personalizzati, e Samsung è l'unica azienda che produce chip e display OLED con il volume di produzione di cui Apple ha bisogno.

La competizione moderna enfatizza la co-evoluzione all'interno degli ecosistemi, dove le organizzazioni collaborano, condividono visioni e formano alleanze per rafforzarsi a vicenda. Come gli ecosistemi naturali, gli ecosistemi aziendali dipendono dall'adattamento continuo, con le relazioni che evolvono per sostenere la vitalità del sistema. Questo è evidente in industrie come quella dei veicoli autonomi, dove le partnership sono cruciali per l'innovazione e il progresso.

\subsubsection{Il Ruolo in Cambiamento della Direzione}

I manager negli ecosistemi aziendali devono andare oltre le strategie aziendali tradizionali e le strutture gerarchiche per costruire reti di partnership. Ciò implica collaborare con partner esterni per affrontare le sfide, accedere alle risorse e promuovere l'innovazione, dimostrando l'importanza delle alleanze strategiche per raggiungere il successo.

Negli ecosistemi aziendali moderni, i manager devono dare priorità ai processi orizzontali e alle relazioni collaborative rispetto alle strutture verticali tradizionali. Questo ruolo di leadership più ampio implica favorire la cooperazione con fornitori, clienti e altri contributori per rafforzare l'ecosistema più grande. Una collaborazione efficace richiede nuove competenze da parte dei dirigenti, poiché i ruoli operativi tradizionali con autorità diretta sono diversi dai ruoli collaborativi, che si basano sulla flessibilità, sulla comunicazione e sull'impegno proattivo per ottenere risultati senza controllo diretto.

La gestione tradizionale si concentrava su ruoli operativi, mantenendo i confini organizzativi e il controllo diretto delle risorse. Tuttavia, i ruoli collaborativi sono diventati essenziali per il successo, poiché la fiducia e le relazioni forti spesso determinano il successo delle partnership più delle strategie o dei piani. Nelle alleanze efficaci, i partner lavorano insieme senza soluzione di continuità, condividendo responsabilità e competenze per raggiungere obiettivi comuni.

\subsubsection{Struttura Interorganizzativa}

Comprendere l'ecosistema organizzativo più ampio è fondamentale per la teoria organizzativa, spostando il ruolo del manager dal controllo verticale alla coordinazione orizzontale tra le organizzazioni. Le relazioni interorganizzative possono essere analizzate attraverso quattro prospettive chiave:

\begin{enumerate}
    \item \textbf{Teoria della dipendenza dalle risorse}: Si concentra sulla minimizzazione della dipendenza dall'ambiente.
    \item \textbf{Reti collaborative}: Evidenzia la dipendenza reciproca per aumentare il valore e la produttività.
    \item \textbf{Ecologia delle popolazioni}: Esamina come le nuove organizzazioni occupano nicchie e diversificano le forme.
    \item \textbf{Istituzionalismo}: Esplora come le organizzazioni acquisiscono legittimità adottando le pratiche di altre.
\end{enumerate}

Questi modelli offrono ai manager gli strumenti per valutare l'ambiente e sviluppare strategie efficaci.

\subsubsection{Reti Collaborative}

La prospettiva delle reti collaborative enfatizza la cooperazione tra organizzazioni per migliorare la competitività, condividere risorse e stimolare l'innovazione. Le aziende formano alleanze per unire conoscenze, soddisfare le esigenze dei clienti e raggiungere obiettivi comuni, richiedendo manager abili nel costruire reti personali forti tra i confini organizzativi. Questo approccio sottolinea il valore della collaborazione nell'affrontare sfide complesse e creare benefici condivisi.

\subsubsection{Perché la Collaborazione}

L'interesse per la collaborazione interorganizzativa è aumentato man mano che le aziende si rendono conto dei vantaggi delle relazioni reciprocamente dipendenti. Le alleanze strategiche aiutano le organizzazioni a condividere i rischi, ridurre i costi, stimolare l'innovazione, adattarsi e migliorare le performance. La collaborazione è particolarmente cruciale per l'ingresso nei mercati globali, dove le joint venture sono fondamentali. Sebbene le aziende nordamericane storicamente abbiano valorizzato l'indipendenza e la competizione, hanno adottato approcci internazionali, come quelli giapponesi e coreani, dimostrando che collaborazione e competizione possono coesistere, portando a un successo maggiore.

I legami interorganizzativi favoriscono investimenti a lungo termine, la condivisione di informazioni e la presa di rischi, consentendo livelli più elevati di innovazione e performance. Spostandosi da una mentalità avversaria a una partnership, le organizzazioni possono ottenere un successo maggiore. La collaborazione tra industrie, anche tra concorrenti, sta diventando più comune, con le partnership che migliorano l'efficienza, l'innovazione e gli obiettivi condivisi, come la creazione di prodotti migliori o il progresso della ricerca. Questa tendenza evidenzia l'importanza crescente della cooperazione per ottenere benefici reciproci e affrontare sfide complesse.

\subsection{Progettazioni per Tecnologie di Produzione e Servizio}

Questo capitolo esplora il ruolo della tecnologia sia nel contesto dei servizi che nella produzione. La tecnologia include i processi, le tecniche, la macchinaria e le azioni che trasformano gli input (materiali, informazioni, idee) in output (prodotti e servizi). Comprende le procedure di lavoro e le attrezzature ed è centrale nel processo produttivo di un'organizzazione. Un tema chiave è come la tecnologia di base modelli la progettazione organizzativa, offrendo spunti su come strutturare le organizzazioni per ottenere performance efficienti. La tecnologia di base si riferisce ai processi chiave allineati con la missione dell'organizzazione, svolgendo un ruolo cruciale nella trasformazione degli input in output.

\subsubsection{Le Imprese Manifatturiere}

Joan Woodward, una sociologa industriale britannica, condusse uno studio pionieristico sulla tecnologia manifatturiera negli anni '50, mettendo in discussione i principi di gestione prevalenti del "metodo migliore unico". Analizz\`o 100 imprese manifatturiere, raccogliendo dati sulle strutture organizzative, gli stili di gestione, i processi produttivi e il successo. Woodward svilupp\`o una scala per classificare le imprese in base alla complessit\`a tecnica dei loro processi di produzione, che riflette il livello di meccanizzazione. Un'elevata complessit\`a indica una forte dipendenza dalle macchine, mentre una bassa complessit\`a implica un maggiore coinvolgimento dei lavoratori. La sua scala fu successivamente semplificata in tre gruppi tecnologici:

\begin{itemize}
    \item \textbf{Gruppo I: Produzione su piccola scala e unit\`a} \newline
    Questo metodo prevede la produzione di ordini personalizzati e di piccole quantit\`a per soddisfare esigenze specifiche del cliente. \`E caratterizzato da un basso livello di meccanizzazione e si basa fortemente sugli operatori umani.
    
    \item \textbf{Gruppo II: Produzione in grandi lotti e di massa} \newline
    Questo metodo comprende lunghe serie di produzione di parti standardizzate, spesso immagazzinate per ordini futuri. \`E pi\`u meccanizzato rispetto alla produzione su piccola scala.
    
    \item \textbf{Gruppo III: Produzione a processo continuo} \newline
    Questo tipo di produzione \`e completamente meccanizzato e standardizzato, operando senza interruzioni. Sistemi automatizzati controllano il processo, producendo risultati altamente prevedibili con un minimo coinvolgimento umano.
\end{itemize}

La ricerca di Woodward rivel\`o che la struttura organizzativa dipende dalla complessit\`a tecnologica. Una maggiore complessit\`a richiede pi\`u gestione e personale di supporto. La produzione di massa \`e altamente standardizzata e meccanicistica, mentre la produzione su unit\`a e i processi continui sono pi\`u flessibili e organici. Diverse tecnologie impongono esigenze uniche, richiedendo adattamenti strutturali.


\subsubsection{Strategia, Tecnologia e Performance}

Woodward ha anche valutato il successo delle imprese basandosi su fattori come la redditività, la quota di mercato, il prezzo delle azioni e la reputazione. Sebbene misurare l'efficacia sia complesso, ha categorizzato le imprese come sopra la media, nella media o sotto la media nel raggiungimento degli obiettivi strategici.

\subsubsection{L'Esplosione dell'Informazione Digitale}

I media tradizionali, come il telefono, la televisione, la radio e i primi computer, richiedevano l'intervento umano. Tuttavia, i media digitali consentono ora alle macchine di creare, modificare e distribuire informazioni in modo autonomo.

L'evoluzione dell'IT è iniziata con i sistemi mainframe progettati per migliorare l'efficienza operativa. Questi sistemi, noti come Transaction Processing Systems (TPS), automatizzavano compiti aziendali di routine come la gestione delle vendite e dell'inventario, archiviando i dati in database per decisioni in tempo reale e servizio clienti.

I computer mainframe hanno portato alla nascita del data warehousing e degli strumenti di business intelligence, consentendo alle organizzazioni di archiviare e analizzare grandi insiemi di dati per decisioni strategiche più informate. Il data mining, componente centrale della business intelligence, analizza dati provenienti da più fonti per identificare tendenze, permettendo ai rivenditori di rispondere rapidamente ai cambiamenti del mercato.

L'avvento dei mini-computer e dei computer desktop ha decentralizzato l'informatica, dando potere agli individui con postazioni di lavoro personali, strumenti di comunicazione e accesso ai database.

L'internet cablata ha rivoluzionato l'IT fornendo accesso globale alle informazioni, mentre l'internet mobile ha accelerato ulteriormente questo processo, abilitando l'analisi dei big data. Ciò ha permesso alle aziende di gestire e analizzare grandi insiemi di dati per decisioni più efficaci.

Le nuove tecnologie di rilevamento aiutano le aziende a raccogliere, cercare e analizzare dati complessi, rivelando schemi che i sistemi tradizionali non possono gestire. Aziende come Walmart e Facebook sfruttano questa tecnologia per migliorare il processo decisionale e mirare meglio la pubblicità.

L'Internet delle Cose (IoT) rappresenta un importante progresso, in cui oggetti di uso quotidiano generano e condividono dati. Questi dispositivi possono comunicare tra loro, raccogliere informazioni e automatizzare risposte. Se combinato con l'intelligenza artificiale, l'IoT offre approfondimenti preziosi e controllo, come il rilevamento di cadute o l'invio di promemoria per i farmaci.


\subsubsection{Pipeline vs Piattaforme}

Le organizzazioni tradizionali, spesso denominate organizzazioni "pipeline", seguono un processo lineare in cui le risorse vengono acquisite, i prodotti vengono realizzati e successivamente venduti ai clienti. Questo modello è stato dominante in settori come la manifattura, i media e l'educazione.

L'ascesa delle organizzazioni basate su piattaforme, guidata da internet e tecnologia mobile, segna un cambiamento significativo. A differenza delle organizzazioni pipeline, le piattaforme non producono beni, ma facilitano le connessioni tra utenti che creano e consumano valore. Esempi come YouTube e Airbnb collegano produttori indipendenti (come videomaker e proprietari di case) con consumatori (come spettatori e turisti), creando un ecosistema digitale.

Le piattaforme sono modelli di business che abilitano scambi tra consumatori e produttori, e talvolta anche gruppi aggiuntivi come sviluppatori di app. Costruiscono ampie reti di utenti e risorse, offrendo accesso on-demand.

Le piattaforme di successo riducono i costi di scambio e non dipendono dalle catene di approvvigionamento per il controllo dell'inventario, come fanno le imprese tradizionali. La differenza chiave è che, nelle organizzazioni basate su piattaforme, il valore risiede nelle connessioni che esse favoriscono, piuttosto che nella proprietà di prodotti o risorse.


\subsubsection{Due Tipi di Piattaforme}

Le organizzazioni basate su piattaforme generalmente rientrano in due categorie: piattaforme di scambio e piattaforme di creazione.

\begin{itemize}
    \item \textbf{Piattaforme di Scambio}: facilitano interazioni 1:1 tra consumatori e produttori, in cui un singolo consumatore effettua una transazione con un singolo produttore.
    \item \textbf{Piattaforme di Creazione}: consentono interazioni 1:1000, in cui i produttori creano contenuti o prodotti che possono essere consumati da molti consumatori contemporaneamente.
\end{itemize}

Entrambi i tipi si basano su connessioni e transazioni digitali piuttosto che su prodotti o servizi fisici. Piattaforme come Amazon sono iniziate come piattaforme di creazione, ma si sono evolute in piattaforme di scambio permettendo ad altre aziende di vendere attraverso la loro infrastruttura, facilitando connessioni digitali tra produttori e consumatori.

\subsubsection{Assunzioni Fondamentali}

Le organizzazioni tradizionali tendono a essere più lente e ingombranti rispetto alle organizzazioni basate su piattaforme, che richiedono meno asset, facilitano una comunicazione più rapida e consentono decisioni più rapide e oggettive. Tuttavia, anche le organizzazioni su piattaforma necessitano di una struttura gerarchica per gestire aspetti critici dell'attività, in particolare la cultura aziendale, che le piattaforme digitali non possono modellare da sole. La leadership e la gerarchia sono essenziali per mantenere la giusta cultura all'interno di un'organizzazione su piattaforma, poiché senza una direzione chiara, l'azienda potrebbe fallire.

\textbf{Esempio: Uber} \newline
La rapida crescita di Uber ha dato priorità all'espansione rispetto alla cultura aziendale, portando a un ambiente di lavoro tossico, reazioni normative e scandali pubblici. Segnalazioni di molestie, inerzia del reparto risorse umane e cattiva condotta della leadership hanno danneggiato la sua reputazione. L'affidamento agli algoritmi per le operazioni ha alienato i conducenti e impedito la creazione di norme culturali solide. Cambiamenti nella leadership hanno successivamente rifocalizzato gli sforzi sul miglioramento della cultura aziendale, dei valori e della reputazione, sottolineando l'importanza di gestire attivamente la cultura organizzativa.

\subsubsection{Raccomandazioni per la Progettazione delle Piattaforme}

Le organizzazioni basate su piattaforme sono ancora in evoluzione e la ricerca sul loro impatto sulla progettazione organizzativa è limitata. Sebbene non esistano linee guida rigide, sono emerse diverse raccomandazioni per progettare efficacemente queste organizzazioni. I leader dovrebbero concentrarsi sulla creazione di una cultura aziendale positiva, investire nel talento dei dipendenti e sviluppare competenze trasversali.

\begin{itemize}
    \item \textbf{Immaginare una Cultura Costruttiva}: I CEO e i leader devono comunicare la loro visione sia per la cultura aziendale che per gli aspetti digitali. Nelle aziende basate su piattaforme, le considerazioni tecniche non dovrebbero oscurare la cultura aziendale. Le trasformazioni digitali di successo spesso si allineano con culture che valorizzano agilità, propensione al rischio, collaborazione e decisioni basate sui dati. I leader dovrebbero stabilire proattivamente norme e valori culturali, guidando lo sviluppo tecnologico invece di lasciarlo definire la cultura.
    
    \item \textbf{Investire nel Talento Digitale}: Acquisire e trattenere talenti digitali è essenziale, specialmente durante periodi di carenza di manodopera. Le organizzazioni dovrebbero coltivare una cultura che incoraggi la sperimentazione e supporti i dipendenti nello sviluppo delle competenze digitali. Molti lavoratori preferiscono la formazione interna, rendendo le aziende che investono nella crescita delle competenze più attraenti. Le organizzazioni digitalmente mature offrono maggiori risorse per lo sviluppo dei talenti, aiutando a mantenere i dipendenti motivati e migliorando le loro competenze digitali.
    
    \item \textbf{Promuovere Competenze Trasversali e Teamwork}: Oltre alle competenze tecniche, le aziende dovrebbero dare priorità allo sviluppo di competenze trasversali come comunicazione, adattabilità e mentalità orientata al cambiamento. Queste sono fondamentali per il successo nel mondo del lavoro digitale. Promuovere la collaborazione e offrire opportunità per il lavoro di squadra tra team diversi accelera i risultati e favorisce l'innovazione. Un forte spirito di squadra e la flessibilità aiutano le organizzazioni ad adattarsi più efficacemente alle trasformazioni digitali.
\end{itemize}

La collaborazione interfunzionale è cruciale per il successo negli ambienti digitali, poiché promuove il lavoro di squadra e l'efficienza organizzativa. Molte aziende ora danno priorità alle competenze trasversali e alle capacità di collaborazione nei processi di assunzione, assicurandosi che i dipendenti possano operare efficacemente in team multidisciplinari. Supportare l'apprendimento continuo e la partecipazione a piattaforme digitali esterne aiuta i dipendenti ad acquisire nuove competenze e rimanere coinvolti. Le organizzazioni che investono nello sviluppo continuo dei talenti sono più propense a trattenere i dipendenti e a rimanere competitive.


\subsection{Analisi dei Big Data}

I big data si riferiscono a enormi insiemi di dati che superano le capacità di elaborazione dei sistemi IT tradizionali, richiedendo nuovi approcci alla gestione e all'analisi dei dati. L'analisi dei big data consiste nell'esaminare questi grandi insiemi di dati per individuare modelli, correlazioni e intuizioni utili a supportare decisioni migliori. A causa della loro dimensione e complessità, gli strumenti tradizionali risultano spesso inadeguati, portando allo sviluppo di nuove tecnologie per l'elaborazione dei big data.

\textbf{Esempio: Siemens}
Siemens Gamesa utilizza l'analisi dei big data su oltre 10.000 turbine eoliche per prevedere e prevenire guasti. Analizzando 200 GB di dati giornalieri provenienti dai sensori di ciascuna turbina, l'azienda pianifica la manutenzione in modo proattivo, riducendo i tempi di inattività, prolungando la vita delle turbine e aumentando la redditività.

\subsection{L'Era Digitale e il Valore dei Big Data}
Nell'era digitale, aziende e clienti attribuiscono sempre più valore ai big data, rendendoli un'opportunità di business significativa. I sensori vengono installati per misurare elementi precedentemente non monitorati, generando enormi quantità di dati utili. Questi dati possono essere raccolti, analizzati e sincronizzati con dashboard per ottenere intuizioni, migliorare il processo decisionale e ottimizzare il monitoraggio di abitudini o sistemi.

\subsection{Requisiti dei Big Data}
\textbf{VOLUME}: I big data sono vasti e la quantità di informazioni generate quotidianamente è enorme. La maggior parte dei dati esistenti è stata creata di recente e la loro crescita è esponenziale. I dati vengono costantemente raccolti, spesso senza che gli individui ne siano consapevoli, attraverso il tracciamento di azioni, posizioni e interazioni. Questa "datafication of everything" rende difficile l'estrazione di intuizioni significative. Sebbene l'archiviazione dei dati sia economica, la vera sfida è la loro analisi efficace.

\textbf{UTILIZZO DI TUTTI I DATI}: I big data permettono alle aziende di analizzare e conservare l'intero set di dati, rivelando correlazioni precedentemente non individuabili. I manager con una mentalità orientata ai big data danno valore a queste correlazioni, in quanto spesso conducono a intuizioni strategiche. I big data possono prevedere i comportamenti dei clienti, consentendo alle aziende di prendere decisioni migliori e ridurre le perdite.

\textbf{UTILIZZO DI DATI IMPERFETTI}: Con l'aumento della quantità di dati, cresce anche la possibilità di imprecisioni, ma nei big data la perfezione non è essenziale. I big data sono spesso disordinati, di qualità variabile e provenienti da fonti diverse. Tuttavia, l'uso di tutti i dati disponibili riduce gli errori rispetto ai metodi di campionamento. I manager accettano imperfezioni poiché queste portano a correlazioni preziose, aiutando le aziende a individuare tendenze e formulare raccomandazioni, migliorando così il processo decisionale.

\textbf{ADOZIONE DI UNA NUOVA MENTALITÀ}: L'approccio ai big data richiede un cambiamento di mentalità che dà priorità ai dati rispetto all'intuizione e all'esperienza tradizionale. Le decisioni vengono basate sull'analisi statistica e sulle intuizioni piuttosto che su informazioni passate o istinti. Questo cambiamento ha creato tensioni tra i manager orientati ai dati e quelli che si affidano all'intuizione. Molti sostenitori dei big data, frustrati dalla resistenza al cambiamento, hanno fondato aziende che utilizzano l'analisi dei dati come strumento principale per il processo decisionale.

\subsection{Big Data e Struttura Organizzativa}
L'outsourcing dell'analisi dei dati è una soluzione comune per le aziende prive di competenze interne, offrendo risparmi sui costi e trasformando i costi fissi in variabili. Consente alle aziende di eseguire progetti rapidamente, ottenere intuizioni e formare i dipendenti. Un'altra opzione è rappresentata dagli intermediari di dati, che aggregano informazioni da più organizzazioni per fornire previsioni di manutenzione o tendenze dei consumatori.

\textbf{Modello Centralizzato}: Consolida gli esperti di dati in un unico dipartimento, garantendo accesso alle competenze e ai dati, ma rischia di limitare l'applicazione su scala aziendale se confinato a una funzione specifica. Un Chief Data Officer (CDO) che riferisce direttamente al CEO può contribuire ad allineare gli sforzi con gli obiettivi aziendali. Alcune aziende, come Intel e IBM, adottano leader per la trasformazione digitale e collaborazioni strategiche per gestire le iniziative di analisi.

\textbf{Modello Ibrido}: Combina approcci centralizzati e decentralizzati, con un piccolo "centro di eccellenza" guidato da un CDO e altri data scientist distribuiti nei vari dipartimenti funzionali. Questo modello favorisce la collaborazione e lo sviluppo di competenze, garantendo al contempo l'allineamento con la strategia aziendale.

\textbf{Modello Decentralizzato}: Incorpora i data scientist direttamente nei dipartimenti, offrendo soluzioni personalizzate, ma limitando la collaborazione tra unità e l'innovazione su scala aziendale. Aziende come Siemens preferiscono la decentralizzazione, anche se può presentare sfide in termini di risorse e competenze. Alcune organizzazioni hanno abbandonato la decentralizzazione in favore della centralizzazione per migliorare l'efficienza e promuovere l'innovazione a livello aziendale.


\subsection{Intelligenza Artificiale}

L'intelligenza artificiale (IA) viene rapidamente adottata sia nelle tecnologie di produzione che nella gestione organizzativa. L'IA migliora il processo decisionale, spesso eguagliando o superando le capacità umane, e automatizza compiti di routine come contabilità, fatturazione, pagamenti e servizio clienti. Può scansionare documenti, verificare registri, inserire dati, elaborare pagamenti e rilevare frodi nei conti spese. Queste applicazioni si concentrano principalmente sull'automazione di compiti ripetitivi, consentendo ai lavoratori di dedicarsi ad attività più significative. Con il tempo, i sistemi di IA continueranno a migliorare grazie all'apprendimento automatico.

Unilever utilizza algoritmi di IA per ottimizzare il processo di selezione del personale. I candidati completano domande online, partecipano a giochi per valutare le loro competenze e inviano risposte video, con l'IA che gestisce la selezione iniziale. Annunci su piattaforme come Facebook attraggono i candidati, e l'IA valuta fattori come concentrazione, memoria, espressioni facciali e tempi di risposta, riducendo il numero di candidati da esaminare. La decisione finale di assunzione prevede un colloquio umano, portando a un processo di selezione più rapido, accurato e con alti tassi di accettazione.

\subsection{La Gestione a Spinte Sarà il Tuo Nuovo Coach?}

I programmi di coaching basati sull'IA utilizzano la gestione a spinte (\textit{nudge management}), che applica principi della scienza comportamentale per incoraggiare comportamenti desiderati attraverso suggerimenti o promemoria. L'obiettivo è migliorare le decisioni relative alla salute, al benessere finanziario, alla felicità e al raggiungimento degli obiettivi. Basata sulle ricerche di Richard Thaler, la gestione a spinte riconosce che le persone spesso scelgono ciò che è più facile rispetto a ciò che è nel loro migliore interesse. Spinte ben tempistiche possono aiutare gli individui a prendere decisioni migliori, ridurre le distrazioni e migliorare le interazioni sul posto di lavoro. Il software può identificare schemi di comportamento e regolare le impostazioni per supportare abitudini più produttive, mentre i promemoria incoraggiano i manager ad adottare azioni in linea con i loro obiettivi e valori.

\subsection{Il Controllo Algoritmico Sarà il Tuo Nuovo Capo?}

Quando il lavoro è misurabile e ripetitivo, il controllo algoritmico utilizza software per impostare obiettivi, misurare le prestazioni, fornire feedback e determinare premi per i dipendenti. Questo livello di controllo traccia anche i dettagli più piccoli delle azioni dei lavoratori, spesso lasciando poco spazio alla privacy. I sistemi di IA monitorano metriche come tassi di produttività, movimenti e interazioni, con dati in tempo reale spesso trasmessi ai supervisori. Il comportamento dei dipendenti, inclusa la produttività e la comunicazione, può essere monitorato attraverso dispositivi, badge o software che analizzano e-mail, utilizzo di internet e battute sulla tastiera. Nei casi più estremi, l'IA può persino valutare e licenziare dipendenti senza l'intervento umano, sollevando preoccupazioni sugli effetti disumanizzanti di tali sistemi. Sebbene le aziende affermino di fornire supporto e sviluppo professionale, alcuni dipendenti riportano di sentirsi eccessivamente sorvegliati e trattati come macchine sotto questi sistemi automatizzati.

\subsection{Implicazioni dell'IA per la Progettazione Organizzativa}

La creazione di un ruolo di \textit{Chief Artificial Intelligence Officer} (CAIO) può aiutare le organizzazioni a implementare l'intelligenza artificiale in modo più efficace. Il CAIO opera trasversalmente tra i dipartimenti per identificare opportunità legate all'IA, definire strategie, coordinare progetti e reclutare esperti nel settore. Questo ruolo è essenziale per gestire i cambiamenti dirompenti introdotti dall'IA, garantendone l'integrazione con la visione a lungo termine dell'organizzazione.

Man mano che l'IA assume compiti di routine, il lavoro non di routine sarà probabilmente gestito da team decentralizzati e flessibili con le competenze necessarie. Questi team richiederanno una supervisione manageriale ridotta, poiché i dipendenti spesso comprendono il lavoro meglio dei loro superiori. Il lavoro non di routine è più efficace quando gestito con maggiore autonomia, e i dipendenti potrebbero essere organizzati secondo una struttura di \textit{team holacracy} piuttosto che in una gerarchia tradizionale.

\subsection{Analisi delle Reti Sociali}

L'\textit{Analisi delle Reti Sociali} (SNA, \textit{Social Network Analysis}) è una tecnica che aiuta i manager a comprendere le relazioni informali e le strutture di rete all'interno di un'organizzazione. Fornisce importanti informazioni sull'influenza, il flusso di conoscenza, l'innovazione e le capacità di leadership, spesso mettendo in evidenza le differenze tra le gerarchie formali e i modelli di comunicazione effettivi.

Nelle reti sociali, gli individui svolgono ruoli chiave come \textit{hub}, \textit{broker} o \textit{attori periferici}. Gli \textbf{hub} sono figure centrali con vasta conoscenza e influenza, rendendoli risorse cruciali per l'organizzazione. I \textbf{broker} connettono gruppi diversi, facilitando la condivisione delle conoscenze e promuovendo la collaborazione tra confini organizzativi. Riconoscere questi ruoli e comprendere le reti informali può conferire un vantaggio competitivo, identificando i principali contributori alle prestazioni, alla collaborazione e all'innovazione.

L'SNA utilizza dati e statistiche per rivelare relazioni nascoste nell'ambiente lavorativo, aiutando a individuare inefficienze e migliorare il lavoro di squadra. Può evidenziare lacune nella comunicazione, suggerire modifiche nei modelli di networking e promuovere la condivisione della conoscenza.

\subsection{Gestione della Conoscenza}

L'analisi delle reti sociali (SNA) può migliorare la gestione della conoscenza di un'organizzazione, aiutando a organizzare, condividere e utilizzare il capitale intellettuale in modo sistematico. La gestione della conoscenza favorisce l'apprendimento continuo e la collaborazione, supportata da strumenti come gli \textit{intranet}, che migliorano la comunicazione, abilitano la condivisione delle informazioni e incentivano il lavoro collaborativo all'interno dell'organizzazione.

Le organizzazioni devono facilitare il trasferimento sia della conoscenza \textbf{codificata} che di quella \textbf{tacita}. La conoscenza \textbf{codificata} è formale, facilmente documentabile e accessibile, mentre la conoscenza \textbf{tacita}, basata sull'esperienza personale e sull'intuizione, è più difficile da articolare ma estremamente preziosa. Poiché gran parte della conoscenza aziendale è tacita, il suo trasferimento è cruciale, soprattutto con il pensionamento dei dipendenti esperti.

Per prevenire la perdita della conoscenza tacita e incoraggiare la condivisione continua del sapere, le organizzazioni adottano due approcci principali: 
\begin{itemize}
    \item Utilizzare sistemi IT per raccogliere e condividere la conoscenza codificata.
    \item Sfruttare la conoscenza tacita attraverso la creazione di reti personali e l'interazione sui social media.
\end{itemize}

I sistemi IT possono anche aiutare a individuare e connettere esperti per facilitare lo scambio di conoscenze, contribuendo a migliorare la competitività e l'efficienza organizzativa.

\subsection{Impatto Digitale sul Design Organizzativo}

I manager e i teorici dell'organizzazione studiano da tempo l'impatto della tecnologia sul design e sul funzionamento organizzativo. I recenti progressi nell'IT digitale hanno condotto a organizzazioni più piccole, decentralizzate, con una migliore coordinazione e nuove strutture a rete. Si prevede che questi sviluppi influenzeranno notevolmente il design organizzativo nei prossimi anni.

\begin{enumerate}
  \item \textbf{Organizzazioni più piccole}: L'IT digitale permette alle organizzazioni di operare con meno personale, conducendo a strutture più ridotte. Le imprese basate su Internet, come i modelli a piattaforma, operano spesso senza uffici fisici. Anche le aziende tradizionali ne traggono beneficio, poiché i sistemi digitali gestiscono compiti amministrativi e permettono transazioni a distanza. L'outsourcing riduce ulteriormente la necessità di risorse interne.
  
  \item \textbf{Strutture decentralizzate}: L'IT digitale consente strutture organizzative decentralizzate facilitando una condivisione delle informazioni rapida e semplice su grandi distanze. La tecnologia promuove apertura, collaborazione e gerarchie meno rigide, permettendo ai manager delle diverse divisioni di prendere decisioni in autonomia, senza attendere l'approvazione della sede centrale. Le tecnologie per il social business supportano la comunicazione e il processo decisionale all'interno di team distribuiti.
  
  \item \textbf{Coordinamento orizzontale potenziato}: L'IT digitale migliora il coordinamento orizzontale, la comunicazione e la collaborazione tra le organizzazioni. Collega persone in tutto il mondo, migliorando il lavoro di squadra e la risoluzione dei problemi, in particolare attraverso team virtuali e strumenti di collaborazione. Queste tecnologie abbattano le barriere tradizionali, come la posizione geografica e le gerarchie, permettendo una condivisione della conoscenza e una collaborazione sui progetti più efficaci.
  
  \item \textbf{Strutture a rete}: L'IT digitale permette strutture a rete migliorate, come organizzazioni virtuali e framework modulari, facilitando il flusso continuo di informazioni tra le aziende. L'outsourcing è diventato più fattibile poiché le organizzazioni possono connettersi facilmente con partner globali, gestire le informazioni e monitorare la produzione in tempo reale. Questa struttura consente alle organizzazioni di ridurre i costi pur espandendo le proprie attività e la presenza sul mercato.
\end{enumerate}


\section{Il Ruolo della Cultura Organizzativa}

Il leader iniziale influenza la cultura dell'organizzazione attraverso la sua visione, filosofia o strategia. Quando queste idee portano al successo, diventano istituzionalizzate, plasmando l'identità dell'organizzazione. Ad esempio, la cultura di Amazon riflette i valori del fondatore Jeff Bezos, che incoraggia i leader a mettere in discussione decisioni o opinioni con cui non concordano, promuovendo una cultura di dissenso aperto, coraggio e convinzione.

La cultura organizzativa svolge due funzioni principali:

\begin{enumerate}
  \item \textbf{Integrazione interna}: Comprende la creazione di un'identità collettiva all'interno dell'organizzazione, guidando la collaborazione, la comunicazione, i comportamenti accettabili e le dinamiche di potere. Assicura che i dipendenti siano allineati con gli obiettivi dell'organizzazione e lavorino bene insieme.
  
  \item \textbf{Adattamento esterno}: La cultura aiuta l'organizzazione a rispondere alle sfide esterne e ad adattarsi all'ambiente, ad esempio soddisfacendo le esigenze dei clienti o mantenendo la competitività sul mercato. Influenza le decisioni e le azioni quotidiane dei dipendenti.
\end{enumerate}

La cultura di un'organizzazione influenza il processo decisionale dei dipendenti in assenza di regole o politiche formali. Inoltre, contribuisce alla costruzione del capitale sociale, modellando relazioni positive o negative sia all'interno che all'esterno dell'organizzazione.

\subsection{Interpretare e Modellare la Cultura Organizzativa}

Comprendere la cultura organizzativa richiede l'interpretazione di artefatti osservabili—elementi che rappresentano e plasmano i valori e le credenze più profonde di un'organizzazione. Questi artefatti, come riti, cerimonie, storie, simboli, strutture, dinamiche di potere e sistemi di controllo, forniscono indizi visibili sulla cultura sottostante e possono essere utilizzati per analizzarla o influenzarla efficacemente. Tuttavia, decifrare accuratamente questi artefatti spesso richiede conoscenze interne ed esperienza, poiché il loro significato può essere profondamente radicato nelle pratiche quotidiane dell'organizzazione.

\subsection{Elementi Chiave della Cultura Organizzativa}

\begin{enumerate}
    \item \textbf{Riti e Cerimonie}: Eventi pianificati che evidenziano e rafforzano i valori dell'organizzazione, creando unità e comprensione condivisa.

    \item \textbf{Storie e Modi di Dire}:
    \begin{itemize}
        \item \textbf{Storie}: Forniscono un contesto alle credenze culturali e aiutano a rafforzare le norme condivise.
        \item \textbf{Modi di dire}: Servono come promemoria semplici di ciò che è importante per l'organizzazione.
    \end{itemize}

    \item \textbf{Simboli}: Elementi sia fisici che astratti che rappresentano valori organizzativi più profondi. Possono includere l'arredamento, la disposizione degli uffici, i loghi o persino rituali che enfatizzano determinati valori.

    \item \textbf{Strutture Organizzative}: Il design dell'organizzazione, inclusa la sua gerarchia, riflette i valori culturali in gioco.
    \begin{itemize}
        \item \textbf{Strutture Rigide}: Possono indicare una cultura che valorizza il controllo e la centralizzazione.
        \item \textbf{Strutture Flessibili e Più Piatte}: Tendono a riflettere una cultura che valorizza l'autonomia e la collaborazione.
    \end{itemize}

    \item \textbf{Relazioni di Potere}: Le dinamiche di potere mostrano come viene esercitata l'influenza all'interno dell'organizzazione, sia attraverso l'autorità formale che tramite fonti informali come competenze o relazioni personali.

    \item \textbf{Sistemi di Controllo}: Si riferiscono a come le operazioni e i dipendenti vengono gestiti e regolamentati, inclusi i sistemi di gestione delle informazioni e le strutture di ricompensa.
\end{enumerate}

\subsection{Cultura a Due Livelli}
\begin{itemize}
    \item \textbf{Valori di Base}: Le convinzioni e le ipotesi più profonde, spesso inconsce, che guidano il comportamento.
    \item \textbf{Artefatti Visibili}: Elementi osservabili come quelli sopra elencati, che possono essere utilizzati per interpretare e influenzare la cultura dell'organizzazione.
\end{itemize}

Questi artefatti non solo riflettono la cultura organizzativa, ma la modellano attivamente. Analizzandoli, i manager possono comprendere meglio le dinamiche culturali esistenti e utilizzarli come strumenti per promuovere o modificare la cultura all'interno dell'organizzazione.

\subsection{Cultura e Design Organizzativo}
Le culture organizzative si concentrano su due dimensioni specifiche:
\begin{enumerate}
    \item Il grado in cui l'ambiente competitivo richiede \textbf{flessibilità} o \textbf{stabilità}.
    \item Il grado in cui il focus strategico dell'organizzazione è \textbf{interno} o \textbf{esterno}.
\end{enumerate}


\subsection{La Cultura dell'Adattabilità}

La cultura dell'adattabilità enfatizza un focus strategico sull'ambiente esterno, dando priorità alla flessibilità. Promuove valori che permettono all'organizzazione di individuare e rispondere ai segnali ambientali con nuovi comportamenti, guidando attivamente il cambiamento piuttosto che limitarsi a reagire. Innovazione, creatività e propensione al rischio sono apprezzate e premiate. Questa cultura è comune tra le aziende basate su Internet, come Google, dove rispondere rapidamente alle esigenze dei clienti è cruciale.

\subsection{La Cultura del Risultato}

La cultura del risultato si concentra sul servizio a clienti specifici in un ambiente stabile, senza la necessità di cambiamenti rapidi. Sottolinea il raggiungimento degli obiettivi, la crescita o la redditività per realizzare la propria visione. I dipendenti sono responsabili delle loro prestazioni, con ricompense legate al raggiungimento di obiettivi specifici. I manager influenzano i comportamenti fissando traguardi e valutando le prestazioni in base a questi. Le culture del risultato danno grande importanza anche alla competitività.

\textbf{Esempio: Huawei} → I dipendenti sono incoraggiati a lavorare molte ore, perseverando anche in condizioni difficili per ottenere nuovi affari. I nuovi dipendenti ricevono una formazione approfondita. Tuttavia, dopo alcune attività illecite all'interno dell'azienda, il CEO ha iniziato a imporre più regole interne.

\subsection{La Cultura del Clan}

La cultura del clan enfatizza il lavoro di squadra, concentrandosi sul soddisfacimento delle esigenze dei dipendenti per raggiungere alte prestazioni. Valorizza la soddisfazione e la produttività dei dipendenti, assicurandosi che abbiano le risorse e il supporto necessari. Questo approccio mira a creare un ambiente positivo e di supporto per i dipendenti, il che a sua volta favorisce il successo dell'organizzazione.

\textbf{Esempio: Southwest Airlines} → “Dipendenti felici portano a clienti felici, che portano a azionisti felici.”

\subsection{La Cultura Burocratica}

La cultura burocratica si concentra sulla stabilità interna e sulla coerenza, supportando un approccio metodico al business. Enfatizza la cooperazione, la tradizione e l'aderenza a politiche e pratiche consolidate per raggiungere gli obiettivi. Sebbene il coinvolgimento personale sia minore, vi è un alto livello di coerenza, conformità e collaborazione tra i membri. Questa cultura ha successo attraverso l'integrazione e l'efficienza.

\subsection{Forza della Cultura e Sottoculture Organizzative}

La \textbf{forza della cultura} si riferisce al grado di accordo tra i membri dell'organizzazione sull'importanza di determinati valori. Una cultura forte è coesa, con aspettative chiare, un ampio consenso sui valori e una bassa tolleranza alla deviazione. Spesso è rafforzata da cerimonie, simboli e storie. I manager allineano strutture e processi per supportare questi valori, favorendo l'impegno dei dipendenti verso gli obiettivi organizzativi. Le culture forti tendono a resistere al cambiamento, poiché i membri valorizzano la stabilità e preferiscono mantenere lo status quo.

La cultura all'interno di un'organizzazione non è sempre uniforme, soprattutto nelle grandi aziende, dove possono svilupparsi \textbf{sottoculture} all'interno di team, dipartimenti o unità per affrontare problemi, obiettivi o esperienze specifiche. Queste sottoculture possono emergere in aree con esigenze diverse, riflettendo valori che differiscono dalla cultura dominante. Ad esempio, mentre la cultura generale può enfatizzare il risultato, alcuni dipartimenti possono adottare valori di adattabilità, clan o burocrazia a seconda delle loro funzioni. Le sottoculture sono modellate dalle esigenze specifiche di ogni unità, come la flessibilità nella ricerca e sviluppo rispetto all'ordine nella produzione.

Le sottoculture generalmente condividono i valori fondamentali della cultura dominante, ma sviluppano anche valori unici. Queste differenze possono causare conflitti tra dipartimenti, specialmente nelle organizzazioni con una cultura aziendale debole. Quando i valori delle sottoculture superano quelli della cultura aziendale, possono influenzare negativamente le prestazioni. Una delle cause più comuni del fallimento delle fusioni aziendali è la difficoltà di integrare culture aziendali differenti.

\subsection{Cultura e Performance}

I manager svolgono un ruolo cruciale nel modellare la cultura organizzativa affinché sia allineata agli obiettivi strategici, poiché la cultura influenza significativamente la performance. Le ricerche dimostrano che le aziende che gestiscono intenzionalmente i propri valori culturali tendono a ottenere risultati migliori rispetto a quelle che non lo fanno. Una cultura forte, che promuove reattività e cambiamento, può migliorare le prestazioni motivando i dipendenti, unendoli attorno a obiettivi condivisi e allineando i comportamenti con le priorità strategiche.

La giusta cultura è essenziale per ottenere alte prestazioni. Le aziende di successo sono quelle in cui i manager vengono valutati e premiati non solo per le loro performance aziendali, ma anche per il loro impegno nel promuovere i valori culturali.

L’approccio tradizionale dell’azienda aveva prodotto risultati finanziari solidi, ma si basava su controllo, intimidazione e un piccolo gruppo esclusivo di dipendenti per motivare il personale. Al contrario, le organizzazioni di \textbf{Quadrante B} rappresentano una cultura ad alte prestazioni, guidata da una chiara missione organizzativa, da valori adattivi condivisi che orientano le decisioni e le pratiche e da un coinvolgimento attivo dei dipendenti nella realizzazione dei risultati aziendali e nella promozione dei valori culturali dell’azienda.


\section{Fondamenti del Comportamento di Gruppo}

Gli individui appartengono a vari gruppi in base a caratteristiche come occupazione, razza e genere, che influenzano la loro percezione delle situazioni. L'identificazione con un gruppo può portare a prospettive diverse, come simpatizzare con le vittime o sostenere le forze dell'ordine, specialmente in situazioni controverse. Le tensioni, come quelle tra le comunità afroamericane e le forze dell'ordine, spesso derivano da conflitti storici e percezioni di pregiudizio. Sebbene i gruppi possano avere influenze positive, possono anche rafforzare i bias. Questo capitolo e il successivo esplorano le dinamiche di gruppo e di team, fornendo strumenti per comprendere e costruire unità di lavoro efficaci.

\subsection{Definizione e Classificazione dei Gruppi}

Nel comportamento organizzativo, un \textbf{gruppo} è definito come due o più individui che interagiscono e sono interdipendenti, riunendosi per raggiungere obiettivi specifici. I gruppi possono essere classificati come \textbf{formali} o \textbf{informali}.

\textbf{Gruppo formale}: creato dalla struttura organizzativa, con compiti e ruoli specifici volti al raggiungimento degli obiettivi aziendali. I ruoli e i comportamenti dei membri sono esplicitamente definiti.

\textbf{Gruppo informale}: non è strutturato formalmente né determinato dall'organizzazione. Questi gruppi si formano naturalmente per soddisfare bisogni di interazione sociale e possono influenzare il comportamento e le prestazioni individuali, anche in assenza di ruoli o compiti formali.

\subsection{Identità Sociale}

Le persone spesso sentono una forte connessione con i propri gruppi, poiché le esperienze condivise, specialmente quelle dolorose, amplificano la fiducia e il legame. Questo attaccamento spiega perché gli individui investano emotivamente nei successi e nei fallimenti del proprio gruppo. Ad esempio, i tifosi di una squadra vincente provano gioia, mentre quelli di una squadra perdente si sentono scoraggiati, nonostante non abbiano avuto un ruolo diretto nel risultato. Questo fenomeno è spiegato dalla \textbf{teoria dell'identità sociale}, che suggerisce che le persone integrano i successi del gruppo nella propria immagine di sé.

La teoria dell'identità sociale afferma che le persone legano la propria autostima al successo o al fallimento del gruppo. Il successo aumenta l'autostima, mentre il fallimento può portare a sentimenti di rifiuto o a sforzi per ristabilire lo status del gruppo. Le minacce all'identità di un gruppo possono provocare reazioni emotive intense, inclusa la soddisfazione per le disgrazie dei gruppi rivali. Gli individui sviluppano molteplici identità sociali basate su fattori come professione, etnia o genere, la cui importanza cambia nel tempo e nei diversi contesti.

Le identità sociali ci aiutano a comprendere noi stessi e il nostro posto nella società, contribuendo a una migliore salute e a una minore depressione riducendo le autoattribuzioni negative. Affinché abbiano effetti benefici, le identità sociali devono essere percepite positivamente.

Nel contesto lavorativo, le identità sociali emergono attraverso due meccanismi:
\begin{itemize}
    \item \textbf{Identificazione relazionale}: connessioni basate sui ruoli.
    \item \textbf{Identificazione collettiva}: connessioni alle caratteristiche del gruppo.
\end{itemize}
L'identificazione può avvenire con team, gruppi di lavoro o intere organizzazioni, ma è spesso più forte nei gruppi di lavoro. Una forte identificazione porta a atteggiamenti e comportamenti positivi, mentre una bassa identificazione può ridurre la soddisfazione, il senso di cittadinanza organizzativa e l'interesse verso organizzazioni che non si allineano alla propria identità collettiva.

\subsection{Ingroup e Outgroup}

L’\textbf{ingroup favoritism} si verifica quando percepiamo i membri del nostro gruppo come superiori agli altri e vediamo i membri esterni al gruppo come omogenei. Le persone con bassa apertura mentale o scarsa amicalità sono più inclini a questo bias.

L'esistenza di un ingroup crea inevitabilmente un outgroup, spesso portando a ostilità tra i due. La religione è un importante fattore scatenante delle dinamiche ingroup-outgroup. I gruppi fortemente coinvolti in rituali e discussioni religiose tendono a mostrare maggiore discriminazione e aggressività verso gli outgroup, specialmente se questi ultimi possiedono più risorse.


\subsection{Minaccia all'Identità Sociale}

L'esistenza di ingroup e outgroup può portare alla \textbf{minaccia all'identità sociale}, in cui gli individui temono di essere giudicati negativamente a causa della loro appartenenza a un gruppo svalutato. Questa minaccia può ridurre la fiducia e le prestazioni. Le ricerche dimostrano che gli individui che sperimentano la minaccia all'identità sociale possono recuperare fiducia e migliorare le loro prestazioni se ricevono incoraggiamento sulle proprie capacità in anticipo.

\subsection{Fasi dello Sviluppo dei Gruppi}

I gruppi temporanei con scadenze definite seguono il \textbf{modello dell'equilibrio punteggiato}. Modelli alternativi descrivono lo sviluppo del gruppo attraverso le seguenti fasi:
\begin{itemize}
    \item \textbf{Forming}: fase iniziale in cui il gruppo si costituisce.
    \item \textbf{Storming}: fase di risoluzione dei conflitti e delle differenze tra i membri.
    \item \textbf{Norming}: fase di accordo sui ruoli e sulle aspettative.
    \item \textbf{Performing}: fase di lavoro collaborativo e produttivo.
\end{itemize}
Queste fasi spesso coincidono con la prima e la seconda fase del modello dell'equilibrio punteggiato, con ulteriori adattamenti delle norme di gruppo tra una fase e l'altra.

Il modello dell'equilibrio punteggiato descrive le fasi dei gruppi temporanei con scadenze definite:

\begin{enumerate}
    \item \textbf{Primo incontro}: Il gruppo stabilisce il proprio scopo generale e la direzione da seguire. Vengono fissati modelli di comportamento e assunzioni, che rimangono invariati durante la prima metà della vita del gruppo. Questo porta a uno stato di \textbf{inerzia}, in cui i progressi sono lenti e i modelli difficili da modificare.
    
    \item \textbf{Transizione a metà percorso}: Si verifica esattamente a metà tra il primo incontro e la scadenza del gruppo. Questa fase funge da segnale d'allarme che aumenta l'urgenza del lavoro. La transizione innesca un cambiamento significativo, portando all'abbandono di vecchi schemi e all'adozione di nuove prospettive. Viene definita una direzione rivista per la seconda fase.
    
    \item \textbf{Seconda fase di inerzia}: Dopo la transizione, il gruppo entra in un nuovo equilibrio, concentrandosi sull'attuazione dei piani elaborati durante la fase di transizione.
    
    \item \textbf{Incontro finale}: Il gruppo attraversa un ultimo periodo di intensa attività per completare il progetto entro la scadenza.
\end{enumerate}

In sintesi, il modello suggerisce che i gruppi attraversano lunghi periodi di inerzia, seguiti da brevi momenti di cambiamento significativo, spesso guidati dalla consapevolezza del tempo e delle scadenze. Sebbene non sia applicabile universalmente, il modello è particolarmente rilevante per i gruppi temporanei che lavorano sotto vincoli di tempo.

\subsection{Proprietà del Gruppo 1: Ruoli}

I gruppi di lavoro influenzano il comportamento individuale e le prestazioni complessive del gruppo. Le proprietà fondamentali di un gruppo includono ruoli, norme, status, dimensione, coesione e diversità.

I \textbf{ruoli} si riferiscono ai modelli di comportamento attesi associati a posizioni specifiche all'interno di un gruppo. Gli individui spesso ricoprono più ruoli sia nella vita professionale che personale, il che può talvolta generare conflitti. Comprendere il comportamento di una persona richiede il riconoscimento del ruolo che essa sta attualmente interpretando. Le persone si basano sulla loro \textbf{percezione del ruolo} per comprendere i comportamenti attesi e soddisfare le aspettative del gruppo.

\subsubsection{Percezione del Ruolo}

La \textbf{percezione del ruolo} è il modo in cui comprendiamo il comportamento atteso in una determinata situazione. Sviluppiamo queste percezioni da diverse fonti, come amici, media o esperienze reali. Ad esempio, gli apprendistati consentono ai principianti di osservare esperti per apprendere il modo corretto di comportarsi in un determinato ruolo.

\subsubsection{Aspettative di Ruolo}

Le \textbf{aspettative di ruolo} si riferiscono a come gli altri credono che una persona debba comportarsi in un determinato contesto. Nel contesto lavorativo, queste aspettative sono definite dal \textbf{contratto psicologico}, un accordo non scritto tra dipendenti e datori di lavoro che delinea le aspettative reciproche. 

\begin{itemize}
    \item I datori di lavoro sono tenuti a fornire un trattamento equo, buone condizioni di lavoro, aspettative chiare e feedback.
    \item I dipendenti, invece, sono tenuti a mostrare un atteggiamento positivo, seguire le direttive e dimostrarsi leali all'organizzazione.
\end{itemize}

Quando questo contratto viene violato, possono verificarsi effetti negativi, come riduzione delle prestazioni, aumento dell'intenzione di lasciare il lavoro e minore soddisfazione, con conseguente calo della produttività e maggiori problemi organizzativi.

\subsubsection{Conflitto di Ruolo}

Il \textbf{conflitto di ruolo} si verifica quando il soddisfacimento di un requisito di ruolo rende difficile adempiere a un altro, portando talvolta a aspettative contraddittorie. Questo può avvenire anche tra diversi gruppi, causando un \textbf{conflitto inter-ruolo}, come la tensione tra ruoli lavorativi e familiari. Il conflitto lavoro-famiglia è una fonte significativa di stress per molti dipendenti.

All'interno delle organizzazioni, i dipendenti spesso affrontano aspettative contrastanti derivanti da ruoli diversi, come l'occupazione, il gruppo di lavoro o il gruppo demografico. Questo fenomeno può essere particolarmente problematico durante fusioni aziendali o all'interno di aziende multinazionali, dove i dipendenti possono avere difficoltà a identificarsi sia con l'organizzazione originale sia con la nuova società madre o divisione globale.

\subsubsection{Interpretazione e Assimilazione del Ruolo}

Il grado con cui ci conformiamo alle aspettative di ruolo, anche quando inizialmente non siamo d'accordo con esse, può essere sorprendente. Un esperimento di \textbf{Philip Zimbardo} ha dimostrato come i partecipanti assegnati ai ruoli di "guardia" e "prigioniero" si adattassero rapidamente alle loro posizioni. Le guardie svilupparono comportamenti abusivi, mentre i prigionieri divennero passivi e sottomessi. Questo esperimento evidenzia come l'\textbf{identità sociale} e la \textbf{percezione del ruolo} possano influenzare fortemente il comportamento.

Le visioni stereotipate dei ruoli, modellate dalle esperienze passate, hanno permesso ai partecipanti di eseguire comportamenti estremi coerenti con i loro ruoli. Tuttavia, un esperimento successivo con minore intensità e una maggiore consapevolezza dell'osservazione esterna ha mostrato che i partecipanti hanno esercitato un maggiore autocontrollo, suggerendo che ambienti meno estremi possono limitare gli abusi di ruolo.

\subsection{Proprietà del Gruppo 2: Norme}

I gruppi stabiliscono \textbf{norme}, ovvero standard condivisi di comportamento accettabile che guidano ciò che i membri dovrebbero o non dovrebbero fare in situazioni specifiche. Una volta accettate dal gruppo, queste norme influenzano il comportamento con un controllo esterno minimo. Sebbene i leader possano esprimere le loro opinioni, la loro influenza è spesso temporanea a meno che il gruppo nel suo insieme non adotti e rafforzi le norme. Le norme variano tra gruppi, comunità e società diverse e possono persino influenzare le emozioni.

\subsubsection{Norme e Conformismo}

Gli individui desiderano essere accettati dal gruppo, il che li rende suscettibili al conformismo rispetto alle norme del gruppo. Studi come gli esperimenti di \textbf{Solomon Asch} dimostrano come la pressione del gruppo possa influenzare le persone a conformarsi nei giudizi e nei comportamenti, anche quando sanno che il gruppo ha torto. Nei suoi esperimenti, il 75\% dei partecipanti si è conformato almeno una volta, anche se il 63\% ha fornito risposte indipendenti nel complesso, dimostrando che il conformismo non è inevitabile. Sebbene gli individui sentano la pressione a conformarsi, spesso preferiscono l'indipendenza e resistono al conformismo quando possibile.

Gli individui non si conformano a tutte le pressioni di ogni gruppo a cui appartengono, poiché le norme variano e possono entrare in conflitto tra loro.

\subsubsection{Norme e Comportamento}

Le norme influenzano significativamente il comportamento del gruppo, come dimostrato nell'ambiente lavorativo e studiato ampiamente durante gli \textbf{Hawthorne Studies} (1924–1932). Questi studi hanno rivelato che la produttività era influenzata più dalla dinamica del gruppo e dall'attenzione percepita che dalle condizioni fisiche. I lavoratori in piccoli gruppi osservati miglioravano le loro prestazioni perché si sentivano "speciali". Allo stesso tempo, gli schemi di incentivi salariali hanno mostrato che le norme del gruppo spesso determinavano i livelli di produttività. I dipendenti aderivano a queste norme per evitare conseguenze negative, come cambiamenti negli incentivi o reazioni negative da parte dei colleghi, anche quando ciò significava lavorare al di sotto del proprio potenziale.

\subsubsection{Norme Positive e Risultati di Gruppo}

Le organizzazioni mirano ad allineare il comportamento dei dipendenti con norme positive per amplificarne l'impatto. Le ricerche suggeriscono che norme positive forti, come quelle relative alla correttezza politica, possono migliorare la creatività del gruppo riducendo l'incertezza e favorendo l'espressione delle idee. Tuttavia, l'efficacia delle norme positive dipende da altri fattori, come le caratteristiche del gruppo (es. estroversione), la personalità individuale, l'identità sociale e la soddisfazione del gruppo, che influenzano l'adesione a tali norme.

\subsubsection{Norme Negative e Risultati di Gruppo}

Il \textbf{comportamento deviante sul lavoro}, o \textbf{comportamento controproduttivo} (CWB, Counterproductive Work Behavior), si riferisce ad azioni volontarie che violano norme organizzative importanti e danneggiano l'organizzazione o i suoi membri. Questi comportamenti includono la diffusione di voci, aggressioni verbali e molestie, che compromettono un ambiente di lavoro sano.

Questi comportamenti emergono spesso in ambienti caratterizzati da tratti negativi, come alti livelli di psicopatia e aggressività nei gruppi di lavoro. L'aumento dell'\textbf{inciviltà sul posto di lavoro}, segnata da scortesia e mancanza di rispetto, porta a maggiore turnover, stress psicologico, malattie fisiche e comportamenti devianti in risposta. Fattori come la percezione di ingiustizia e la mancanza di sonno, spesso dovuti a pressioni lavorative elevate, contribuiscono a tali comportamenti, indicando che le pressioni organizzative possono inavvertitamente favorire la devianza.

Il comportamento deviante è influenzato dalle norme del gruppo e prospera in ambienti in cui tali comportamenti sono tollerati. Questo porta a una riduzione della cooperazione, dell'impegno e della motivazione. Nei gruppi disfunzionali, i comportamenti devianti possono innescare una reazione a catena, causando umori negativi, scarsa coordinazione e un calo delle prestazioni.

\subsubsection{Norme e Cultura}

Le norme variano tra le culture \textbf{collettiviste} e \textbf{individualiste}, ma le tendenze individuali possono cambiare in base alla situazione. Le ricerche mostrano che quando gli individui sono esposti a scenari che enfatizzano norme collettiviste o individualiste, la loro motivazione aumenta se i compiti sono in linea con l'orientamento ricevuto. Ad esempio:
\begin{itemize}
    \item La scelta personale aumenta la motivazione nelle culture individualiste.
    \item L'assegnazione a un gruppo interno stimola la motivazione nelle culture collettiviste.
\end{itemize}
Questo dimostra che le norme culturali influenzano il comportamento e la motivazione degli individui in contesti organizzativi.

\subsection{Proprietà del Gruppo 3: Status}

Lo \textbf{status}, ovvero il rango o la posizione socialmente definita all'interno di un gruppo, esiste in tutte le società e si sviluppa anche nei piccoli gruppi nel tempo. Lo status influenza fortemente il comportamento, soprattutto quando gli individui percepiscono una discrepanza tra il proprio status auto-valutato e la percezione che gli altri hanno di esso.

Secondo la \textbf{teoria delle caratteristiche di status}, lo status è determinato da tre fattori chiave:
\begin{enumerate}
    \item \textbf{Potere}: Gli individui che controllano le risorse o i risultati del gruppo sono percepiti come aventi uno status più elevato.
    \item \textbf{Contributi}: Coloro le cui capacità sono essenziali per raggiungere gli obiettivi del gruppo tendono a detenere uno status più alto.
    \item \textbf{Caratteristiche personali}: Attributi come intelligenza, attrattiva, ricchezza o personalità amichevole, se apprezzati dal gruppo, aumentano lo status individuale.
\end{enumerate}

\subsubsection{Status e Norme}

Gli individui di alto status possono influenzare le norme del gruppo e la pressione al conformismo. È più probabile che si discostino dalle norme se hanno una bassa identificazione con il gruppo e sono meno influenzati dai membri di rango inferiore. Poiché sono valorizzati ma non dipendenti dalle ricompense sociali, possono resistere alla pressione del conformismo. Sebbene i membri di alto status possano migliorare le prestazioni del gruppo, possono anche introdurre norme controproducenti.

\subsubsection{Status e Interazione di Gruppo}

Gli individui che mirano a uno status più elevato tendono a diventare più assertivi, parlano di più, criticano, danno ordini e interrompono gli altri. I membri di status inferiore possono partecipare meno, e se le loro competenze non vengono utilizzate, le prestazioni del gruppo possono risentirne. Tuttavia, un equilibrio tra membri di status medio e alto è più vantaggioso: troppi membri di alto status possono ostacolare le prestazioni complessive.

\subsubsection{Disparità di Status}

La \textbf{disparità di status} si verifica quando i membri percepiscono la gerarchia di status come ingiusta, portando a insoddisfazione, prestazioni peggiori, problemi di salute e maggiore propensione al turnover tra i membri di status inferiore. I gruppi concordano sui criteri di status, formando gerarchie informali basate su risorse o influenza. I conflitti possono emergere quando gli individui si spostano tra gruppi con criteri di status diversi o provenienti da contesti culturali differenti, complicando la dinamica del gruppo.

\subsubsection{Status e Stigmatizzazione}

Lo status è influenzato sia dagli attributi personali sia dalle associazioni sociali. La \textbf{stigmatizzazione per associazione} si verifica quando le percezioni negative di una persona stigmatizzata si estendono a coloro che vi sono affiliati, anche in interazioni brevi. Questo può portare a una svalutazione ingiusta, poiché i pregiudizi contro i gruppi stigmatizzati persistono indipendentemente dal merito individuale.

\subsubsection{Status di Gruppo}

Fin dalla giovane età, le persone sviluppano una mentalità di \textit{"noi contro loro"}, in cui i gruppi interni (\textit{ingroup}) spesso detengono uno status più elevato e possono discriminare i gruppi esterni (\textit{outgroup}). I gruppi di status inferiore possono utilizzare il favoritismo verso il proprio gruppo per competere per un status più elevato, portando i gruppi di alto status a percepire discriminazione nei loro confronti. Questo intensifica il pregiudizio e aumenta la polarizzazione tra i gruppi.


\subsection{Proprietà del Gruppo 4: Dimensione e Dinamiche}
La dimensione del gruppo influisce sul comportamento, con i gruppi più numerosi che eccellono nella generazione di input e idee diversificate, mentre i gruppi più piccoli sono più efficaci per compiti produttivi. Una questione chiave nei gruppi più grandi è il \textit{social loafing} (inerzia sociale), in cui gli individui esercitano meno sforzo quando lavorano collettivamente. Ciò accade a causa di percezioni di iniquità, in cui i membri sentono che gli altri non contribuiscono in modo equo, o a causa di una \textit{diffusione della responsabilità}, in cui i contributi individuali sono più difficili da distinguere nel risultato complessivo del gruppo. Il \textit{social loafing} sfida l'assunzione che la produttività di gruppo equivalga alla somma degli sforzi individuali.

Il \textit{social loafing} ha implicazioni significative per il comportamento organizzativo. Per mitigare i suoi effetti, i manager possono impostare obiettivi di gruppo, promuovere la competizione tra i gruppi, implementare valutazioni tra pari, selezionare membri motivati e orientati al gruppo, e legare le ricompense ai contributi individuali. Pubblicare le valutazioni delle performance individuali aiuta anche a contrastare il \textit{social loafing}.

\subsection{Proprietà del Gruppo 5: Coesione e Proprietà del Gruppo 6: Diversità}
\subsubsection{Proprietà del Gruppo 5: Coesione}
I gruppi variano nella loro coesione, che è l'estensione con cui i membri sono attratti tra di loro e motivati a rimanere nel gruppo. La coesione influisce sulla produttività del gruppo, con l'effetto che dipende dalle norme di performance del gruppo. Alta coesione associata a forti norme di performance porta a una maggiore produttività, mentre alta coesione con basse norme porta a una bassa produttività. Quando la coesione è bassa, la produttività dipende dalla forza delle norme, ma in generale è meno efficace rispetto ai gruppi ad alta coesione con forti norme.

Per promuovere la coesione del gruppo, si può:
\begin{enumerate}
    \item Ridurre la dimensione del gruppo.
    \item Promuovere l'accordo sugli obiettivi del gruppo.
    \item Aumentare il tempo che i membri trascorrono insieme.
    \item Favorire la competizione con altri gruppi.
    \item Fornire ricompense al gruppo nel suo complesso, piuttosto che agli individui.
\end{enumerate}

\subsection{Proprietà del Gruppo 6: Diversità}
La diversità di gruppo si riferisce al grado di differenze tra i membri, offrendo sia vantaggi che svantaggi. Può aumentare il conflitto, in particolare nelle prime fasi, abbassando il morale e aumentando i tassi di abbandono. Mentre gruppi diversi e omogenei possono avere prestazioni simili, i gruppi diversi tendono ad avere una minore soddisfazione, coesione e maggiore conflitto. L'impatto della diversità dipende dalla gestione del gruppo e dalle pratiche delle risorse umane, con una leadership efficace che aiuta a ridurre il conflitto. La diversità di genere può essere una sfida, ma promuovere l'inclusività può alleviare i problemi. Il tipo di diversità è importante: la diversità di superficie spesso segnala differenze più profonde e incoraggia la mentalità aperta, mentre la diversità di livello profondo può sia migliorare le prestazioni che ridurre la competizione non produttiva.

La diversità può creare conflitto, ma offre anche opportunità uniche per la risoluzione dei problemi. I gruppi diversi possono affrontare sfide iniziali, ma possono diventare più aperti, creativi e ottenere prestazioni migliori nel tempo. L'impatto complessivo della diversità è misto e i suoi effetti positivi non sono sempre forti. Il caso aziendale per la diversità, in particolare in termini finanziari, è difficile da dimostrare. Nei team diversi, specialmente con diversità di superficie, possono emergere le \textit{faultlines}—divisioni percepite basate sulle differenze individuali—che separano il gruppo in sottogruppi.

La ricerca sulle \textit{faultlines} indica che le divisioni all'interno dei gruppi danneggiano tipicamente le prestazioni, portando a competizione, apprendimento più lento, decisioni più rischiose, ridotta creatività e maggiore conflitto. Sebbene i sottogruppi possano sentirsi soddisfatti, la soddisfazione complessiva del gruppo tende ad essere più bassa. Tuttavia, le \textit{faultlines} basate su abilità, conoscenze ed expertise possono giovare alle organizzazioni orientate ai risultati. Se ai sottogruppi viene dato un obiettivo comune e sono incoraggiati a collaborare, gli effetti negativi delle \textit{faultlines} possono essere mitigati. In alcuni casi, le \textit{faultlines} relative a caratteristiche pertinenti al compito possono migliorare le prestazioni promuovendo la divisione del lavoro. Pertanto, le \textit{faultlines} possono essere dannose, ma se usate strategicamente, possono migliorare le prestazioni.

\section{Processo Decisionale di Gruppo}
\subsection{Gruppo versus Individuo}
Il processo decisionale di gruppo è comunemente utilizzato nelle organizzazioni, ma se è preferibile rispetto al processo decisionale individuale dipende da vari fattori. Per valutarne l'efficacia, è importante considerare i punti di forza e di debolezza del processo decisionale di gruppo.

\subsubsection{Punti di Forza del Processo Decisionale di Gruppo}
I gruppi offrono informazioni e conoscenze più complete combinando le risorse di più individui, fornendo una diversità di opinioni e potenziali alternative. Questa diversità aumenta le possibilità di considerare approcci differenti. Inoltre, i membri del gruppo sono più propensi ad accettare e supportare una decisione, favorendo una maggiore accettazione all'interno dell'organizzazione.

\subsubsection{Punti di Debolezza del Processo Decisionale di Gruppo}
Le decisioni di gruppo tendono a essere dispendiose in termini di tempo, poiché spesso richiedono più tempo per raggiungere una soluzione. Possono esserci pressioni di conformità, in cui i membri evitano di dissentire per essere accettati dal gruppo. Le discussioni di gruppo possono anche essere dominate da alcuni individui, riducendo l'efficacia complessiva. Inoltre, la responsabilità per la decisione è spesso ambigua, poiché la responsabilità è condivisa, diversamente dalla decisione individuale in cui la responsabilità è chiara.

\subsubsection{Efficacia ed Efficienza}
Il processo decisionale di gruppo è più accurato rispetto alla media degli individui, ma è meno efficiente e più lento. Eccelle nella creatività e nell'accettazione delle soluzioni. Tuttavia, generalmente richiede più tempo rispetto alle decisioni individuali, a meno che una singola persona non abbia bisogno di un ampio input. I manager devono bilanciare l'efficacia delle decisioni di gruppo con la loro inefficienza.

I gruppi sono preziosi per il processo decisionale, poiché offrono input diversificati e un'analisi critica, che migliora la raccolta delle informazioni e il supporto alle soluzioni. Tuttavia, il tempo speso, i conflitti interni e le pressioni di conformità possono ridurne l'efficacia. I conflitti legati al compito possono migliorare le prestazioni, mentre i conflitti relazionali le danneggiano. In alcune situazioni, gli individui possono prendere decisioni migliori rispetto ai gruppi.

\subsection{Groupthink e Groupshift}
Il \textit{groupthink} e il \textit{groupshift} sono due effetti negativi derivanti dal processo decisionale di gruppo che possono ostacolare una valutazione obiettiva e soluzioni di alta qualità.

\subsubsection{Groupthink}
Il \textit{groupthink} si verifica quando le pressioni del gruppo per la conformità impediscono una valutazione critica delle opinioni di minoranza o impopolari, limitando spesso la creatività e l'efficacia. È più probabile che si manifesti in gruppi con una forte identità, un desiderio di proteggere la propria immagine positiva e una minaccia collettiva a tale immagine. I gruppi focalizzati sulle prestazioni piuttosto che sull'apprendimento sono particolarmente suscettibili. Il \textit{groupthink} può essere ridotto controllando la dimensione del gruppo, incoraggiando una leadership imparziale, nominando un avvocato del diavolo per sfidare la maggioranza e utilizzando esercizi che si concentrano sui rischi potenziali prima di discutere i benefici. Queste strategie aiutano a favorire una discussione aperta e un processo decisionale obiettivo.

\subsubsection{Groupshift}
Il \textit{groupshift}, o polarizzazione di gruppo, si verifica quando le discussioni di gruppo portano i membri ad adottare versioni più estreme delle loro posizioni originali. I conservatori diventano più cauti, mentre i prenditori di rischi diventano più aggressivi. Questo spostamento è guidato da fattori come un maggiore comfort nell'esprimere opinioni estreme, la diffusione della responsabilità e il desiderio di differenziarsi dai gruppi esterni. Comprendere la polarizzazione di gruppo aiuta i manager a riconoscere che le decisioni di gruppo tendono ad esagerare le posizioni individuali. La direzione del cambiamento dipende dalle inclinazioni iniziali dei membri. Tecniche per ridurre il processo decisionale disfunzionale di gruppo possono aiutare a mitigare questi effetti.


\subsection{Tecniche di Decisione di Gruppo}
La forma più comune di processo decisionale di gruppo si verifica nei gruppi interattivi, in cui i membri si incontrano faccia a faccia e comunicano sia verbalmente che non verbalmente. Tuttavia, come si è visto con il \textit{groupthink}, questi gruppi a volte possono portare all'autocensura e alla pressione per la conformità. Tecniche come il brainstorming e la \textit{Nominal Group Technique} (NGT) possono aiutare a ridurre questi problemi, incoraggiando input più aperti e diversificati.


Il brainstorming stimola la creatività permettendo l'espressione di tutte le idee, anche quelle non convenzionali, senza critiche. In una sessione tipica, i partecipanti suggeriscono quante più alternative possibile, mentre il leader si assicura che il problema sia chiaro e registra tutte le idee per la discussione successiva. Tuttavia, la ricerca mostra che il brainstorming nei gruppi è meno efficiente rispetto alla generazione di idee individuali a causa del \textit{production blocking}, in cui le discussioni simultanee interrompono i processi mentali degli individui, rallentando la condivisione delle idee.


La \textit{Nominal Group Technique} (NGT) è un approccio più strutturato al processo decisionale di gruppo che limita la discussione e la comunicazione interpersonale. Essa prevede i seguenti passaggi:
\begin{enumerate}
    \item Ogni membro scrive indipendentemente le proprie idee prima di qualsiasi discussione.
    \item I membri presentano le loro idee una alla volta, senza discussione, fino a quando tutte le idee sono state condivise e registrate.
    \item Il gruppo discute le idee per chiarimenti e le valuta.
    \item Ogni membro classifica silenziosamente le idee, e l'idea con la classifica complessiva più alta determina la decisione finale. Questo metodo aiuta a ridurre il \textit{production blocking} e il \textit{groupthink}.
\end{enumerate}

Il principale vantaggio della NGT è che consente riunioni formali di gruppo mantenendo al contempo il pensiero indipendente. La ricerca mostra che i gruppi nominali generalmente ottengono risultati migliori rispetto ai gruppi di brainstorming. Ogni tecnica decisionale di gruppo ha i suoi punti di forza e di debolezza. La scelta della tecnica dipende dai criteri desiderati e dai compromessi tra costi e benefici. I gruppi interattivi sono efficaci per ottenere impegno, il brainstorming favorisce la coesione del gruppo e la NGT genera efficacemente un gran numero di idee.

\section{Evoluzione del ruolo del CISO}

Il ruolo del Chief Information Security Officer (CISO) sta evolvendo da un focus principalmente tecnico a un ruolo di leadership strategica che collega la trasformazione aziendale alla cybersecurity. I CISO moderni devono connettere gli sforzi di cybersecurity con gli obiettivi organizzativi, guidando iniziative che favoriscono il successo.

I CISO efficaci sono esperti sia in ambito aziendale che tecnologico, abili nel collaborare con i dirigenti C-suite, i consigli di amministrazione e nel navigare sfide normative e di privacy complesse. Favoriscono inoltre una prospettiva condivisa del rischio tra gestione del rischio, conformità e leadership. Tuttavia, solo il 26\% delle organizzazioni considera attualmente lo sviluppo di metriche di rischio per la cybersecurity una responsabilità centrale, sebbene questa tendenza rifletta una crescente integrazione della cybersecurity con la gestione del rischio complessiva e la governance dei dati.

Le organizzazioni spesso reclutano i CISO esternamente, sebbene alcune promozioni interne avvengano, in particolare da ruoli tecnologici, di conformità o consulenza. Con l’evolversi del ruolo in una direzione più strategica, la costruzione di una cultura aziendale orientata alla cybersecurity e la formulazione di strategie stanno diventando responsabilità fondamentali. I compiti tecnici di routine, come l'identificazione delle minacce, sono destinati ad essere sempre più automatizzati e meno centrali nel ruolo del CISO.

Le sfide chiave includono la scarsa partecipazione dei CISO nelle fasi iniziali dello sviluppo dei prodotti, con meno della metà coinvolta nei test, nello sviluppo o nella strategia dei prodotti. Ciò limita la capacità dell'organizzazione di incorporare la sicurezza sin dall'inizio, perdendo opportunità di sfruttare l'expertise del CISO.

Per colmare queste lacune, le organizzazioni dovrebbero:
\begin{itemize}
    \item Consentire ai CISO di concentrarsi sulla strategia e sulla promozione di una cultura della cybersecurity.
    \item Separare il ruolo del CISO dall'IT e promuovere un maggiore coinvolgimento con i dirigenti e i consigli di amministrazione.
    \item Coinvolgere i CISO nelle prime fasi dello sviluppo di prodotti e applicazioni.
    \item Investire nel reclutamento, nella formazione e nel fornire risorse adeguate alla leadership della cybersecurity.
    \item Sviluppare un business case solido per programmi di cybersecurity a livello organizzativo.
\end{itemize}

Questa evoluzione posiziona il CISO come un elemento chiave per la resilienza strategica in un ambiente sempre più digitale e ad alto rischio.


\section{NONAKA}
In un'economia in costante cambiamento, la conoscenza è cruciale per mantenere un vantaggio competitivo sostenibile. Le aziende che hanno successo sono quelle che generano costantemente nuova conoscenza, la condividono e la integrano rapidamente nei prodotti e nelle tecnologie. Tuttavia, molti manager faticano a gestire questo processo a causa di una comprensione limitata della conoscenza, modellata dalla focalizzazione della gestione occidentale su dati formali, quantificabili e su metriche di efficienza.

Un approccio più dinamico, visto nelle aziende di successo, si concentra sulla gestione della conoscenza creativa e olistica, dando priorità all'innovazione, all'adattabilità e alla reattività alle esigenze del mercato. Questo approccio garantisce il successo in un ambiente aziendale in rapida evoluzione.

L'approccio giapponese all'innovazione sottolinea la creazione di nuova conoscenza utilizzando le intuizioni tacite e soggettive dei dipendenti, piuttosto che limitarsi a elaborare dati oggettivi. Questo approccio implica un impegno personale, un'identità organizzativa e l'uso di simboli e metafore per incorporare la conoscenza nei prodotti e nelle tecnologie.

Le aziende giapponesi vedono l'organizzazione come un'entità vivente con un senso condiviso di scopo, promuovendo l'innovazione attraverso il rinnovamento continuo e l'allineamento con i valori comuni. La creazione della conoscenza è uno sforzo collettivo, dove ogni dipendente è un lavoratore della conoscenza.

La lezione principale per i manager è mettere la creazione della conoscenza al centro delle strategie delle risorse umane, prendendo spunto dai metodi giapponesi per progettare organizzazioni focalizzate sull'innovazione continua e sull'adattabilità.

Il processo di creazione della conoscenza organizzativa inizia con gli individui che trasformano la loro conoscenza personale e tacita in conoscenza esplicita e condivisa. La conoscenza tacita, profondamente personale e radicata nell'esperienza, comprende sia le competenze tecniche ("know-how") che i quadri cognitivi come i modelli mentali e le credenze. Questo tipo di conoscenza è difficile da formalizzare e comunicare, ma influenza il modo in cui gli individui percepiscono e affrontano i problemi.

L'esempio di Ikuko Tanaka alla Matsushita evidenzia come la conoscenza tacita, come le tecniche di panificazione osservate da un maestro panettiere, possa essere trasformata in conoscenza esplicita, come le specifiche del prodotto, tramite esperimentazione e collaborazione. La conoscenza esplicita, essendo formale e sistematica, è più facile da condividere in tutta l'organizzazione e da utilizzare come base per l'innovazione.

Questo movimento tra conoscenza tacita ed esplicita è al cuore del processo di creazione della conoscenza, consentendo alle aziende di trasformare le intuizioni individuali in preziosi beni organizzativi.

La distinzione tra conoscenza tacita ed esplicita è alla base di quattro modelli chiave di creazione della conoscenza all'interno delle organizzazioni, formando una dinamica "spirale della conoscenza":

\begin{enumerate}
    \item \textbf{Tacita a Tacita (Socializzazione)}: La conoscenza tacita viene condivisa direttamente tramite osservazione, imitazione e pratica, come si vede quando un apprendista impara da un maestro. Tuttavia, questo processo rimane limitato perché la conoscenza resta personale e non può essere facilmente sfruttata dall'organizzazione.
    
    \item \textbf{Esplicita a Esplicita (Combinazione)}: La conoscenza esplicita viene sintetizzata in nuove forme, come un rapporto finanziario che integra dati provenienti da fonti diverse. Sebbene utile, questo processo non espande significativamente la base di conoscenza dell'organizzazione.
    
    \item \textbf{Tacita a Esplicita (Articolazione)}: La conoscenza tacita viene convertita in conoscenza esplicita, rendendola condivisibile attraverso l'organizzazione. Per esempio, un lavoratore articola le proprie intuizioni per creare soluzioni innovative o specifiche di prodotto. Questa interazione è particolarmente potente per la creazione di conoscenza.
    
    \item \textbf{Esplicita a Tacita (Internalizzazione)}: La conoscenza esplicita condivisa viene assorbita dai dipendenti, riplasmando la loro comprensione tacita e le loro competenze. Con il tempo, diventa una parte intuitiva delle loro pratiche lavorative, abilitando ulteriori innovazioni.
\end{enumerate}

Questi processi interagiscono dinamicamente. Ad esempio, Ikuko Tanaka alla Matsushita ha appreso la conoscenza tacita da un panettiere (socializzazione), l'ha articolata in specifiche esplicite del prodotto (articolazione), ha aiutato il team a standardizzarla in un manuale (combinazione) e ha interiorizzato l'esperienza per comprendere intuitivamente gli standard di qualità (internalizzazione).

Questa spirale continua consente alla base di conoscenza dell'organizzazione di crescere ed evolversi, con ogni iterazione che eleva la comprensione e favorisce l'innovazione a un livello superiore.

L'articolazione (convertire la conoscenza tacita in conoscenza esplicita) e l'internalizzazione (usare la conoscenza esplicita per migliorare la comprensione tacita) sono fondamentali nel processo di creazione della conoscenza. Entrambi richiedono un impegno personale e implicano l'espressione della propria visione del mondo, portando alla reinvenzione degli individui, dell'organizzazione e del suo impatto più ampio. Gestire questo processo richiede approcci che differiscono significativamente dalle pratiche di gestione tradizionali occidentali.


\end{document}   