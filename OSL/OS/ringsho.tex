\documentclass{article}
\usepackage[utf8]{inputenc}
\usepackage{enumitem}

\begin{document}

\title{Ringi Sho - Summary}
\author{}
\date{}

\maketitle

\section*{Summary}

\begin{itemize}
    \item \textbf{Definition}:
    \begin{itemize}
        \item \textit{Ringi Sho} is a Japanese collective decision-making process based on consensus, rather than hierarchical decisions.
        \item "Ringi" means "circulation of documents" and "Sho" means "written proposal" or "document."
    \end{itemize}

    \item \textbf{Process}:
    \begin{itemize}
        \item A subordinate prepares the \textit{Ringi Sho} document.
        \item It is circulated through various hierarchical levels for approval or suggestions.
        \item Each participant stamps the document with a personal seal (\textit{Hanko}).
        \item The document gathers approvals or revisions until it reaches the higher management for final approval.
    \end{itemize}

    \item \textbf{Properties}:
    \begin{itemize}
        \item \textbf{Collective participation}:
        \begin{itemize}
            \item Everyone is involved, even those at the lowest hierarchical levels can express their opinions.
        \end{itemize}
        \item \textbf{Gradual process}:
        \begin{itemize}
            \item The document passes through various levels, gathering consensus step by step.
        \end{itemize}
        \item \textbf{Flexibility and transparency}:
        \begin{itemize}
            \item Every approval or comment is visible, ensuring transparency and openness to revision.
        \end{itemize}
        \item \textbf{Long duration}:
        \begin{itemize}
            \item The process can take more time than centralized decision-making but guarantees more involvement.
        \end{itemize}
    \end{itemize}

    \item \textbf{Cultural Reflection}:
    \begin{itemize}
        \item Reflects the strong emphasis on collaboration and consensus in Japanese work culture.
        \item Minimizes internal conflicts but can be less efficient in contexts requiring fast decisions.
    \end{itemize}

    \item \textbf{Positive Aspects}:
    \begin{itemize}
        \item \textbf{Improved organizational cohesion}:
        \begin{itemize}
            \item Involvement in decision-making fosters a sense of belonging and collaboration.
        \end{itemize}
        \item \textbf{Better decision-making}:
        \begin{itemize}
            \item Decisions based on consensus have a higher probability of being accepted and effectively implemented.
        \end{itemize}
    \end{itemize}

    \item \textbf{Criticism}:
    \begin{itemize}
        \item The process is slow and complex.
        \item It is difficult to apply in competitive environments that require fast decisions.
    \end{itemize}
\end{itemize}

\end{document}
